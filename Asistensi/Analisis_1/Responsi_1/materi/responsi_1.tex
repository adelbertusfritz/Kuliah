\documentclass{beamer}
\usepackage[utf8]{inputenc}
\usepackage[T1]{fontenc}
\title{Responsi 1 - Analisis 1}
\subtitle{Bilangan Riil - Part 1}
\date[]{2024/2025}
\author[Fritz]{Fritz Adelbertus Sitindaon}

\usetheme{fritz}
% ===== Setup Font =====
\usepackage[sfdefault,lf]{carlito}
\usepackage[T1]{fontenc}
\renewcommand*\oldstylenums[1]{\carlitoOsF #1}

% ==== Import Math Packages =====
\usepackage{amsmath, amssymb, amsthm}
\usepackage{mathtools}

\def\R{\mathbb{R}}
\def\P{\mathbb{P}}
\def\N{\mathbb{N}}

\begin{document}

\begin{frame}
\titlepage
\end{frame}

\begin{frame}{Outline}
    \tableofcontents
\end{frame}

\begin{frame}{Sifat Aljabar $\R$}
    \begin{tcolorbox}[enhanced,title=Penjumlahan, frame style tile={width=\paperwidth}{\wallpaper}]
        \begin{itemize}
            \item Komutatif: $\forall a,b \in \R, a+b=b+a$.
            \item Asosiatif: $\forall a,b,c \in \R, (a+b)+c=a+(b+c)$.
            \item Unit: $\exists 0 \in R \ni \forall a \in R, a+0=0+a=a$.
            \item Invers: $\forall a \in \R, \exists -a \in R \ni a + (-a) = (-a) + a = 0$.
        \end{itemize}
    \end{tcolorbox}
\end{frame}

\begin{frame}{Sifat Aljabar $\R$}
    \begin{tcolorbox}[enhanced,title=Perkalian, frame style tile={width=\paperwidth}{\wallpaper}]
        \begin{itemize}
            \item Komutatif: $\forall a,b \in \R, a \cdot b=b \cdot a$.
            \item Asosiatif: $\forall a,b,c \in \R, (a \cdot b) \cdot c=a \cdot (b \cdot c)$.
            \item Unit: $\exists 1 \in R \ni \forall a \in R, a \cdot 1=1 \cdot a=a$.
            \item Invers: $\forall a \in \R, \exists 1/a \in R \ni a \cdot 1/a = 1/a \cdot a = 1$.
        \end{itemize}
    \end{tcolorbox}
\end{frame}

\begin{frame}{Sifat Aljabar $\R$}
    \begin{tcolorbox}[enhanced,title=Distributif, frame style tile={width=\paperwidth}{\wallpaper}]
        $\forall a,b,c \in R$
        \begin{itemize}
            \item $a \cdot (b+c) = (a \cdot b) + (a \cdot c)$.
            \item $(b+c) \cdot a = (b \cdot a) + (c \cdot a)$.
        \end{itemize}
    \end{tcolorbox}
\end{frame}

\begin{frame}{Sifat Keterurutan $\R$}
    \begin{tcolorbox}[enhanced,title=Bilangan Positif $\P$, frame style tile={width=\paperwidth}{\wallpaper}]
        \begin{itemize}
            \item Jika $a,b \in \P$, maka $a+b \in \P$.
            \item Jika $a,b \in \P$, maka $a\cdot b \in \P$.
            \item Jika $a \in \R$, maka $a \in \P$ atau $-a \in \P$ atau $a=0$.
        \end{itemize}
    \end{tcolorbox}
    \begin{tcolorbox}[enhanced,title=Definisi 2.1.6, frame style tile={width=\paperwidth}{\wallpaper}]
        \begin{itemize}
            \item $a-b \in P$ dapat ditulis $a > b$ atau $b < a$.
            \item $a-b \in P \cup {0}$ dapat ditulis $a \geq b$ atau $b \geq a$.
        \end{itemize}
    \end{tcolorbox}
\end{frame}

\begin{frame}{Teorema Familiar}
    \begin{tcolorbox}[enhanced,title=Teorema 2.1.7, frame style tile={width=\paperwidth}{\wallpaper}]
        $\forall a,b,c \in \R$
        \begin{itemize}
            \item Jika $a > b$ dan $b > c$, maka $a > c$.
            \item Jika $a > b$ maka $a+c > b+c$.
            \item Jika Jika $a > b$ dan $c > 0$, maka $ca > cb$.\\
            Jika $a > b$ dan $c < 0$, maka $ca < cb$.
        \end{itemize}
    \end{tcolorbox}
    \begin{tcolorbox}[enhanced,title=Teorema 2.1.10, frame style tile={width=\paperwidth}{\wallpaper}]
        Jika $ab > 0$, maka
        \begin{itemize}
            \item $a > 0$ dan $b > 0$, atau
            \item $a < 0$ dan $b < 0$.
        \end{itemize}
    \end{tcolorbox}
\end{frame}

\begin{frame}{Teorema Spesial}
    \begin{tcolorbox}[enhanced,title=Teorema 2.1.9, frame style tile={width=\paperwidth}{\wallpaper}]
        Jika $a \in \R$ dan $\forall \epsilon > 0, 0 \leq a < \epsilon$, maka $a=0$.
    \end{tcolorbox}
    \begin{tcolorbox}[enhanced,title=Ketaksamaan Bernouli, frame style tile={width=\paperwidth}{\wallpaper}]
        Jika $x > -1$, maka $(1+x)^n \geq 1 +nx \quad \forall n \in \N$.
    \end{tcolorbox}
    \begin{itemize}
        \item Teorema 2.1.9 dapat digunakan untuk membuktikan 2 bilangan sama dan sering digunakan 
        di materi selanjutnya.
        \item Ketaksamaan Bernouli suka muncul di pembuktian yang tidak diduga.
    \end{itemize}
\end{frame}

\begin{frame}{Teorema Lain}
    \begin{tcolorbox}[enhanced,title=Teorema 2.1.2 (Terkait Unit), frame style tile={width=\paperwidth}{\wallpaper}]
        \begin{itemize}
            \item Jika $z,a \in \R$ dan $z+a=a$, maka $z=0$.
            \item Jika $u,b \in \R$, $u,b \neq 0$ dan $u \cdot b = b$, maka $u = 1$.
            \item Jika $a \in \R$, maka $a \cdot 0 = 0$.
        \end{itemize}
    \end{tcolorbox}
    \begin{itemize}
        \item Teorema ini menjelaskan bahwa hanya ada satu elemen di $\R$ yang
        menjadi elemen unit dalam operasi penjumlahan dan perkalian.
        \item Poin terakhir memberikan pengecualian untuk nilai $0$.
    \end{itemize}
\end{frame}


\begin{frame}{Teorema Lain}
    \begin{tcolorbox}[enhanced,title=Teorema 2.1.3 (Terkait Invers), frame style tile={width=\paperwidth}{\wallpaper}]
        \begin{itemize}
            \item Jika $a,b \in \R$, $a \neq 0$ dan $a \cdot b = 1$, maka $b = 1/a$.
            \item Jika $a \cdot b = 0$, maka $a = 0$ atau $b = 0$.
        \end{itemize}
    \end{tcolorbox}
    \begin{itemize}
        \item Teorema ini menjelaskan bahwa setiap bilangan di $\R$ punya tepat satu invers terhadap operasi perkalian.
        \item Poin terakhir memberikan pengecualian untuk nilai $0$.
    \end{itemize}
\end{frame}


\begin{frame} 
\frametitle{Sifat Aljabar $\R$} 
\begin{theorem}
There is no largest prime number. \end{theorem} 
\begin{enumerate} 
\item<1-| alert@1> Suppose $p$ were the largest prime number. 
\item<2-> Let $q$ be the product of the first $p$ numbers. 
\item<3-> Then $q+1$ is not divisible by any of them. 
\item<1-> But $q + 1$ is greater than $1$, thus divisible by some prime
number not in the first $p$ numbers.
\end{enumerate}
\end{frame}





\end{document}