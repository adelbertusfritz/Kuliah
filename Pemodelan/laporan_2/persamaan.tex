% ===== Setup Page Layout =====
\documentclass{article}
\usepackage{geometry}
 \geometry{
 a4paper,
 total={15cm, 20cm},
 }
\usepackage{graphicx}
% ===== Setup Font =====
\usepackage[sfdefault,lf]{carlito}
\usepackage[T1]{fontenc}
\renewcommand*\oldstylenums[1]{\carlitoOsF #1}

% ==== Import Math Packages =====
\usepackage{amsmath, amssymb, amsthm}
\usepackage{mathtools}

% ==== Import Styling Packages =====
\usepackage{enumitem}
\usepackage[pages=some, placement=bottom]{background}
\usepackage{moresize}
\usepackage{relsize}
\usepackage{hyperref}
\hypersetup{colorlinks=true,allcolors=blue}
\usepackage{hypcap}
\usepackage{verbatim}
\usepackage[normalem]{ulem}

\usepackage{hyperref}

% ==== Custom Declarations =====
\DeclarePairedDelimiter\abs{\lvert}{\rvert}
\DeclarePairedDelimiter\floor{\lfloor}{\rfloor}
\DeclarePairedDelimiter\cic{[ }{] }
\DeclarePairedDelimiter\oic{( }{] }
\DeclarePairedDelimiter\cio{[ }{) }
\DeclarePairedDelimiter\oio{( }{) }
\DeclarePairedDelimiter\set{\{ }{\} }
\DeclarePairedDelimiter\brk{(}{)}
\newcommand{\drv}[2]{\frac{d}{d#1}\brk*{#2}}
\newcommand{\drvL}[2]{D_{#1}\brk*{#2}}
\newcommand{\ds}{\displaystyle}
\newcommand{\eval}[3]{\left.\brk*{#1}\right\rvert_{#2}^{#3}}
\newcommand{\R}{\mathbb{R}}
\newcommand{\vecu}{\textbf{u}}

\title{PD di COMCOT untuk Pemodelan Tsunami}
\author{Fritz Adelbertus Sitindaon}
\date{}

\begin{document}
\begin{flushright}
    \section*{COMCOT untuk Pemodelan Tsunami}
    \textbf{Fritz Adelbertus Sitindaon}
\end{flushright}
\textbf{PERINGATAN!}\\
ringkasan ini digunakan untuk menjelaskan secara umum proses untuk menjawab pertanyaan berikut:\\
\textbf{Persamaan Differensial yang digunakan COMCOT untuk Memodelkan Gelombang Tsunami di Perairan}

\vspace{0.5cm}\hrule height 2pt\vspace{0.5cm}

\begin{center}
    \textbf{Persamaan Gelombang Tsunami Di Laut Dalam (Koordinat Bola)}
    \\Linear Shallow Water Wave Equation
\end{center}
\begin{equation}
    \frac{\partial \eta}{\partial t} + \frac{1}{R\cos \varphi}\left\{\frac{\partial P}{\partial \psi}+\frac{\partial}{\partial \varphi}(\cos \varphi Q)\right\} = -\frac{\partial h}{\partial t}
\end{equation}
\begin{equation}
    \frac{\partial P}{\partial t} + \frac{gh}{R\cos \varphi}\frac{\partial \eta}{\partial \psi}-fQ=0
\end{equation}
\begin{equation}
    \frac{\partial Q}{\partial t} + \frac{gh}{R}\frac{\partial \eta}{\partial\varphi}+fP=0
\end{equation}

\begin{enumerate}
    \item $\eta:$ Fluktuasi permukaan air $(m)$
    \item $P,Q:$ Volume flux component (volume air yang berpindah pada permukaan terhadap waktu)
    $P$ mengarah ke timur (sumbu-$x$), $Q$ mengarah ke utara (sumbu-$y$). $(m/s)$
    \item $(\varphi, \psi):$, koordinat bola (latitude dan longitude)
    \item $R:$ jari-jari bumi $(m)$
    \item $g:$ gravitasi $(m/s^2)$
    \item $h:$ kedalaman air $(m)$
    \item $-\frac{\partial h}{\partial t}:$ representasikan variasi bathymetry
    \item $f$ gaya Coriolis dengan $f=\Omega \sin\varphi$, $\Omega$ adalah rate rotation dari bumi
    \item $H:$ total kedalaman air $h+\eta$
    \item $F_x, F_y$: gaya gesek bawah laut pada arah sumbu-$x/y$
    \[
        F_x=\frac{gn^2}{H^{7/3}}P\oio*{P^2+Q^2}^{1/2}
    \]
    \[
        F_y=\frac{gn^2}{H^{7/3}}Q\oio*{P^2+Q^2}^{1/2}
    \]
    \item $n:$ koefisien Manning (bergantung ruang)
\end{enumerate}

\[
F_S = F_D + F_{HS}
\]
\[
F_{HS} = \rho gd_S A
\]
\begin{enumerate}
    \item $F_S:$ \textit{Surge Force}
    \item $F_D:$ Gaya gesek
    \item $F_{HS}:$ Gaya hidrostatis
    \item $\rho:$ Massa jenis air laut
    \item $g:$ Gravitasi
    \item $h,d_s:$ ketinggian tsunami
    \item $C_D:$ Koefisien gesek
    \item $A:$ Luas permukaan proyeksi terhadap arah gelombang
    \item $u:$ \textit{tsunami-bore velocity}
    \item $C:$ koefisien konstan
\end{enumerate}


\begin{center}
    \textbf{Persamaan Gelombang Tsunami Di Pesisir (Koordinat Bola)}\\
    Nonlinear Shallow Water Wave Equation
\end{center}
\begin{equation}
    \frac{\partial \eta}{\partial t} + \frac{1}{R\cos \varphi}\left\{\frac{\partial P}{\partial \psi}+\frac{\partial}{\partial \varphi}(\cos \varphi Q)\right\} = -\frac{\partial h}{\partial t}
\end{equation}
\begin{equation}
    \frac{\partial P}{\partial t} + \frac{1}{R\cos\varphi}\frac{\partial}{\partial \psi}\left\{\frac{P^2}{H}\right\}+\frac{1}{R}\frac{\partial}{\partial\varphi}
    \left\{\frac{PQ}{H}\right\} + \frac{gH}{R\cos\varphi}\frac{\partial\eta}{\partial\psi}-fQ+F_x=0
\end{equation}
\begin{equation}
    \frac{\partial Q}{\partial t} + \frac{1}{R\cos\varphi}\frac{\partial}{\partial \psi}\left\{\frac{PQ}{H}\right\}+\frac{1}{R}\frac{\partial}{\partial\varphi}
    \left\{\frac{Q^2}{H}\right\} + \frac{gH}{R}\frac{\partial\eta}{\partial\varphi}+fP+F_y=0
\end{equation}

\begin{enumerate}
    \item $H:$ total kedalaman air $h+\eta$
    \item $F_x, F_y$: gaya gesek bawah laut pada arah sumbu-$x/y$
    \[
        F_x=\frac{gn^2}{H^{7/3}}P\oio*{P^2+Q^2}^{1/2}
    \]
    \[
        F_y=\frac{gn^2}{H^{7/3}}Q\oio*{P^2+Q^2}^{1/2}
    \]
    \item $n:$ koefisien Manning (bergantung ruang)
\end{enumerate}


\begin{center}
    \textbf{Solusi Numerik Persamaan Gelombang Tsunami Di Laut Dalam (Koordinat Bola)}
    \\Explicit Leap-Frog Finite Difference
\end{center}
\begin{align*}
    &\frac{\eta_{i,j}^{n+1/2}-\eta_{i,j}^{n-1/2}}{\Delta t} 
    + \left\{\frac{1}{R\cos \varphi}\right\}_{i,j}\frac{P_{i+1/2,j}^n-P_{i-1/2,j}^n}{\Delta\psi}
\end{align*}
\begin{align}
    &+ \left\{\frac{1}{R\cos \varphi}\right\}_{i,j}\frac{\oio*{\cos\varphi_{i,j+1/2}}Q_{i,j+1/2}^n-\oio*{\cos\varphi_{i,j-1/2}}Q_{i,j-1/2}^n}
    {\Delta \varphi} = -\frac{h_{i,j}^{n+1/2}-h_{i,j}^{n-1/2}}{\Delta t}
\end{align}
\begin{equation}
    \frac{P_{i+1/2,j}^{n+1}-P_{i+1/2,j}^n}{\Delta t} 
    + \left\{\frac{gh}{R\cos\varphi}\right\}_{i+1/2,j}
    \frac{\eta_{i+1,j}^{n+1/2}-\eta_{i,j}^{n+1/2}}{\Delta\psi}
    -fQ_{i,j+1/2}^n=0
\end{equation}
\begin{equation}
    \frac{Q_{i,j+1/2}^{n+1}-Q_{i+1/2,j+1/2}^n}{\Delta t} 
    + \left\{\frac{gh}{R}\right\}_{i,j+1/2}
    \frac{\eta_{i,j+1}^{n+1/2}-\eta_{i,j}^{n+1/2}}{\Delta\varphi}
    +fP_{i+1/2,j}^n=0
\end{equation}

\begin{enumerate}
    \item index $i$ mengacu ke step di sumbu-$x$ (longitud $\Delta\psi$)
    \item index $j$ mengacu ke step di sumbu-$y$ (latitud $\Delta\varphi$)
    \item index $n$ mengacu ke step di waktu ($\Delta t$)
\end{enumerate}


\begin{center}
    \textbf{Solusi Numerik Persamaan Gelombang Tsunami Di Pesisir (Koordinat Bola)}
    \\Upwind Finite Difference
\end{center}
Stability Requirement: $\sqrt{2gh}\Delta t/\Delta x < 1$
\begin{equation}
    \eta_{i,j}^{n+1/2} = \eta_{i,j}^{n-1/2} - \frac{1}{R\cos\varphi_{i,j}}\frac{\Delta t}{\Delta \psi}
    \oio*{P_{i+1/2,j}^n-P_{i-1/2,j}^n}-\frac{1}{R}\frac{\Delta t}{\Delta \varphi}
    \oio*{Q_{i,j+1/2}^n-Q_{i,j-1/2}^n}
\end{equation}
\begin{align*}
    P_{i+1/2,j}^{n+1}=&f_x\left\{(1-v_x\Delta t)P_{i+1/2,j}^{n+1}
    -\left\{\frac{gH}{R\cos\varphi}\right\}_{i+1/2,j}^{n+1/2}\frac{\Delta t}{\Delta \psi}
    \oio*{\eta_{i+1,j}^{n+1/2}-\eta_{i,j}^{n+1/2}}\right\}\\
    &-\frac{f_x}{R\cos\varphi_{i+1/2,j}}\frac{\Delta t}{\Delta \psi}\left\{
    \lambda_{11}\frac{\oio*{P_{i+3/2,j}^n}^2}{H_{i+3/2,j}^n}
    +\lambda_{12}\frac{\oio*{P_{i+1/2,j}^n}^2}{H_{i+1/2,j}^n}
    +\lambda_{13}\frac{\oio*{P_{i-1/2,j}^n}^2}{H_{i-1/2,j}^n}
    \right\}\\
    &-\frac{f_x}{R}\frac{\Delta t}{\Delta \varphi}\left\{
    \lambda_{21}\frac{(PQ)^n_{i+1/2,j+1}}{H_{i+1/2,j+1}^n}
    +\lambda_{22}\frac{(PQ)^n_{i+1/2,j}}{H_{i+1/2,j}^n}
    +\lambda_{23}\frac{(PQ)^n_{i+1/2,j-1}}{H_{i+1/2,j-1}^n}
    \right\}
\end{align*}
\begin{align*}
    Q_{i,j+1/2}^{n+1}=&f_y\left\{(1-v_y\Delta t)Q_{i,j+1/2}^{n+1}
    -\left\{\frac{gH}{R}\right\}_{i,j+1/2}^{n+1/2}\frac{\Delta t}{\Delta \psi}
    \oio*{\eta_{i,j+1}^{n+1/2}-\eta_{i,j}^{n+1/2}}\right\}\\
    &-\frac{f_y}{R\cos\varphi_{i,j+1/2}}\frac{\Delta t}{\Delta \psi}\left\{
    \lambda_{31}\frac{(PQ)^n_{i+1,j+1/2}}{H_{i+1,j+1/2}^n}
    +\lambda_{32}\frac{(PQ)^n_{i,j+1/2}}{H_{i,j+1/2}^n}
    +\lambda_{33}\frac{(PQ)^n_{i-1,j+1/2}}{H_{i-1,j+1/2}^n}
    \right\}\\
    &-\frac{f_y}{R}\frac{\Delta t}{\Delta \varphi}\left\{
    \lambda_{41}\frac{\oio*{Q_{i,j+3/2}^n}^2}{H_{i,j+3/2}^n}
    +\lambda_{42}\frac{\oio*{Q_{i,j+1/2}^n}^2}{H_{i,j+1/2}^n}
    +\lambda_{43}\frac{\oio*{Q_{i,j-1/2}^n}^2}{H_{i,j-1/2}^n}
    \right\}
\end{align*}
Dimana 
\[
f_x=1/(1+v_x\Delta t),\quad f_y=1/(1+v_y\Delta t)
\]
dengan (ini masih belum jelas kenapa ada $Q_{i+1/2,j}$ dan $P_{i,j+1/2}$)
\[
v_x = \frac{1}{2}\frac{gn^2}{\oio*{H^n_{i+1/2,j}}^{7/3}}
\cic*{\oio*{P^n_{i+1/2,j}}^2+\oio*{Q^n_{i+1/2,j}}^2}^{1/2}
\]
\[
v_y = \frac{1}{2}\frac{gn^2}{\oio*{H^n_{i+1/2,j}}^{7/3}}
\cic*{\oio*{P^n_{i,j+1/2}}^2+\oio*{Q^n_{i,j+1/2}}^2}^{1/2}
\]
juga koefisien $\lambda$
\begin{align*}
    &\begin{cases}
        \lambda_{11}=0,\,\lambda_{12}=1,\,\lambda_{13}=-1,&\text{jika $P^n_{i+1/2,j}\geq 0$}\\
        \lambda_{11}=1,\,\lambda_{12}=-1,\,\lambda_{13}=0,&\text{jika $P^n_{i+1/2,j}< 0$}
    \end{cases}\\
    &\begin{cases}
        \lambda_{21}=0,\,\lambda_{22}=1,\,\lambda_{23}=-1,&\text{jika $Q^n_{i+1/2,j}\geq 0$}\\
        \lambda_{21}=1,\,\lambda_{22}=-1,\,\lambda_{23}=0,&\text{jika $Q^n_{i+1/2,j}< 0$}
    \end{cases}\\
    &\begin{cases}
        \lambda_{31}=0,\,\lambda_{32}=1,\,\lambda_{33}=-1,&\text{jika $P^n_{i,j+1/2}\geq 0$}\\
        \lambda_{31}=1,\,\lambda_{32}=-1,\,\lambda_{33}=0,&\text{jika $P^n_{i,j+1/2}< 0$}
    \end{cases}\\
    &\begin{cases}
        \lambda_{41}=0,\,\lambda_{42}=1,\,\lambda_{43}=-1,&\text{jika $Q^n_{i,j+1/2}\geq 0$}\\
        \lambda_{41}=1,\,\lambda_{42}=-1,\,\lambda_{43}=0,&\text{jika $Q^n_{i,j+1/2}< 0$}
    \end{cases}
\end{align*}
Lebih jelas di Manual COMCOT

\vspace{0.5cm}\hrule height 2pt\vspace{0.5cm}

\noindent Asumsi yang disertakan
\begin{enumerate}
    \item inviscous flow, (tidak adanya disipasi energi)
    \item variasi vertikal diabaikan
\end{enumerate}
Rasio grid size 3-5
\vspace{0.5cm}\hrule height 2pt\vspace{0.5cm}

\noindent Referensi: \\
X. Wang; W. Power 2011. COMCOT: a Tsunami Generation Propagation and
Run-up Model, GNS Science Report 2011/43 129 p.

\end{document}
