\section{Pendahuluan}
\begin{frame}{Megathrust}
     
    Megathrust adalah tipe zona penunjaman dengan kedalaman 
    penunjaman sekitar lebih dari 50km. Zona penunjaman 
    merupakan tempat pertemuan/ interaksi antar 
    lempeng, khususnya yang bersifat tumbukan (convergent).
    Gempa bumi 
    bersumber dari megathrust berpotensi menghasilkan 
    gempa bumi dengan kekuatan besar, yaitu magnitudo 
    lebih dari delapan sehingga berpotensi terjadi tsunami. 

\end{frame}
\begin{frame}{Megathrust}
    \begin{center}
        \includegraphics[scale=0.7]{megathrust_region.png}
        \tiny{https://vsi.esdm.go.id/press-release/sumber-gempa-bumi-zona-penunjam-megathrust}
    \end{center}
    
\end{frame}

\begin{frame}{Aktivitas Seismik}
    Aktivitas seismik merujuk pada frekuensi, 
    jenis, dan magnitudo gempa bumi yang terjadi 
    dalam kurun waktu tertentu di suatu wilayah. 
    Aktivitas ini terjadi akibat pergerakan lempeng 
    tektonik di bawah permukaan bumi, yang dapat 
    menyebabkan pelepasan energi dalam bentuk 
    gelombang seismik. Gelombang ini merambat 
    melalui bumi dan menyebabkan permukaan tanah 
    bergetar. 
\end{frame}

\begin{frame}{Gelombang Seismik}
    Gelombang ini terbagi menjadi dua jenis utama: 
    gelombang badan dan gelombang permukaan. 
    Gelombang badan terdiri dari gelombang primer 
    (P-waves) yang bergerak cepat dan gelombang 
    sekunder (S-waves) yang lebih lambat, keduanya 
    merambat melalui bagian dalam Bumi. Sementara 
    itu, gelombang permukaan, seperti gelombang 
    Love dan Rayleigh, bergerak di sepanjang 
    permukaan bumi dan cenderung menyebabkan lebih 
    banyak kerusakan.
\end{frame}
