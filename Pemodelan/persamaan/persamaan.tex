% ===== Setup Page Layout =====
\documentclass{article}
\usepackage{geometry}
 \geometry{
 a4paper,
 total={15cm, 20cm},
 }
\usepackage{graphicx}
% ===== Setup Font =====
\usepackage[sfdefault,lf]{carlito}
\usepackage[T1]{fontenc}
\renewcommand*\oldstylenums[1]{\carlitoOsF #1}

% ==== Import Math Packages =====
\usepackage{amsmath, amssymb, amsthm}
\usepackage{mathtools}

\newtheorem{theorem}{Teorema}[section]
\newtheorem{corollary}{Akibat}[theorem]
\newtheorem{lemma}[theorem]{Lemma}
\newtheorem{definition}[theorem]{Definisi}

% ==== Import Styling Packages =====
\usepackage{enumitem}
\usepackage[pages=some, placement=bottom]{background}
\usepackage{moresize}
\usepackage{relsize}
\usepackage{hyperref}
\hypersetup{colorlinks=true,allcolors=blue}
\usepackage{hypcap}
\usepackage{verbatim}
\usepackage[normalem]{ulem}

\usepackage{hyperref}

% ==== Custom Declarations =====
\DeclarePairedDelimiter\abs{\lvert}{\rvert}
\DeclarePairedDelimiter\floor{\lfloor}{\rfloor}
\DeclarePairedDelimiter\cic{[ }{] }
\DeclarePairedDelimiter\oic{( }{] }
\DeclarePairedDelimiter\cio{[ }{) }
\DeclarePairedDelimiter\oio{( }{) }
\DeclarePairedDelimiter\set{\{ }{\} }
\DeclarePairedDelimiter\brk{(}{)}
\DeclarePairedDelimiter\vct{\langle}{\rangle}
\newcommand{\Mod}[1]{\ (\mathrm{mod}\ #1)}
\newcommand{\drv}[2]{\frac{d}{d#1}\brk*{#2}}
\newcommand{\drvL}[2]{D_{#1}\brk*{#2}}
\newcommand{\ds}{\displaystyle}
\newcommand{\eval}[3]{\left.\brk*{#1}\right\rvert_{#2}^{#3}}
\newcommand{\R}{\mathbb{R}}
\newcommand{\Rubik}{
    \draw[black, thick] (0,0) -- (2.598,1.5);
    \draw[black, thick] (0,0) -- (-2.598,1.5);
    \draw[black, thick] (0,0) -- (0,-3);
    \draw[black, thick] (0,-3) -- (2.598,-1.5);
    \draw[black, thick] (0,-3) -- (-2.598,-1.5);
    \draw[black, thick] (0,-2) -- (2.598,-0.5);
    \draw[black, thick] (0,-2) -- (-2.598,-0.5);
    \draw[black, thick] (0,-1) -- (2.598,0.5);
    \draw[black, thick] (0,-1) -- (-2.598,0.5);
    \draw[black, thick] (2.598,-1.5) -- (2.598,1.5);
    \draw[black, thick] (-2.598,-1.5) -- (-2.598,1.5);
    \draw[black, thick] (0,3) -- (2.598,1.5);
    \draw[black, thick] (0,3) -- (-2.598,1.5);
    \draw[black, thick] (0.867,0.5) -- (0.867,-2.5);
    \draw[black, thick] (1.732,1) -- (1.732,-2);
    \draw[black, thick] (-0.867,0.5) -- (-0.867,-2.5);
    \draw[black, thick] (-1.732,1) -- (-1.732,-2);
    \draw[black, thick] (-0.867,0.5) -- (1.732,2);
    \draw[black, thick] (-1.732,1) -- (0.867,2.5);
    \draw[black, thick] (0.867,0.5) -- (-1.732,2);
    \draw[black, thick] (1.732,1) -- (-0.867,2.5);
}

\newcommand{\RubikU}{
    \draw[black, thick] (0,0) -- (2.598,1.5);
    \draw[black, thick] (0,0) -- (-2.598,1.5);
    \draw[black, thick] (0,3) -- (0,6);
    \draw[black, thick] (0,6) -- (2.598,4.5);
    \draw[black, thick] (0,6) -- (-2.598,4.5);
    \draw[black, thick] (0,4) -- (2.598,2.5);
    \draw[black, thick] (0,4) -- (-2.598,2.5);
    \draw[black, thick] (0,5) -- (2.598,3.5);
    \draw[black, thick] (0,5) -- (-2.598,3.5);
    \draw[black, thick] (2.598,4.5) -- (2.598,1.5);
    \draw[black, thick] (-2.598,4.5) -- (-2.598,1.5);
    \draw[black, thick] (0,3) -- (2.598,1.5);
    \draw[black, thick] (0,3) -- (-2.598,1.5);
    \draw[black, thick] (0.867,5.5) -- (0.867,2.5);
    \draw[black, thick] (1.732,5) -- (1.732,2);
    \draw[black, thick] (-0.867,2.5) -- (-0.867,5.5);
    \draw[black, thick] (-1.732,2) -- (-1.732,5);
    \draw[black, thick] (-0.867,0.5) -- (1.732,2);
    \draw[black, thick] (-1.732,1) -- (0.867,2.5);
    \draw[black, thick] (0.867,0.5) -- (-1.732,2);
    \draw[black, thick] (1.732,1) -- (-0.867,2.5);
}

\title{Persamaan Gelombang untuk Pemodelan Megathrust}
\author{Fritz Adelbertus Sitindaon}
\date{}

\begin{document}
\begin{flushright}
    \section*{Persamaan Gelombang untuk Pemodelan Megathrust}
    \textbf{Fritz Adelbertus Sitindaon}
\end{flushright}
\textbf{PERINGATAN!}\\
ringkasan ini digunakan untuk menjelaskan secara umum proses untuk menjawab pertanyaan berikut:\\
\textbf{Diberikan data yang cukup untuk memodelkan tsunami, bagaimana dampaknya jika tsunami mencapai daratan?}

three models needed:
\begin{enumerate}
    \item tsunami waves propagation modelling
    \item inundation modeling (pembanjiran)
    \item damage assessment on land
\end{enumerate}

\vspace{0.5cm}\hrule height 2pt\vspace{0.5cm}

\begin{center}
    \textbf{Persamaan Gelombang Tsunami (2 Dimensi)}
\end{center}
Dari hukum kekekalan massa:
\[
\frac{\partial h}{\partial t} + \frac{\partial (hu)}{\partial x} + \frac{\partial (hv)}{\partial y} = 0
\]
Dari hukum kekekalan momentum:
\[
\frac{\partial (hu)}{\partial t} + \frac{\partial}{\partial x} \oio*{hu^2+\frac{1}{2}gh^2} + \frac{\partial (huv)}{\partial y} = 0
\]
\[
\frac{\partial (hv)}{\partial t} + \frac{\partial (huv)}{\partial x} + \frac{\partial}{\partial y} \oio*{hu^2+\frac{1}{2}gh^2} = 0
\]
\begin{enumerate}
    \item $h(x,y,t)$ adalah kedalaman air pada titik $(x,y)$ pada waktu $t$.
    \item $u(x,y,t)$ kecepatan horizontal pada sumbu-$x$
    \item $v(x,y,t)$ kecepatan horizontal pada sumbu-$y$
    \item $g$ adalah percepatan dari gravitasi
\end{enumerate}

\noindent\textbf{Requirements Modelling Tsunami Waves}
\begin{enumerate}
    \item Initial information needed:
    \begin{enumerate}
        \item magnitude, yaitu energy yang dihasilkan oleh kejadian seismik
        \item epicenter, koordinat pusat sumber gelombang
        \item kedalaman, terhadap permukaan laut
        \item fault geometry, fault length, width, dip angle, slip
        \item bathymetry data, peta topografi dasar laut beserta kedalamannya
    \end{enumerate}
    \item Model Seafloor Displacement \& Initial Conditions Setup\\
    Gunakan model Okada untuk memperolah kondisi awal yang akan dimasukkan kedalam persamaan gelombang tsunami
    \item Fit to Mathematical Model
    \item Solve (numerically)
    \item Integrate data bathymetry
\end{enumerate}
Hasil: Simulasi model tsunami dari titik terjadi sampai ujung pesisir

\vspace{0.5cm}\hrule height 2pt\vspace{0.5cm}
\begin{center}
    \textbf{Model Pembanjiran}
\end{center}
Persamaan gelombang tsunami sebelumnya dapat diperluas ke darat\\
\noindent\textbf{Requirements Modelling Inundation}
\begin{enumerate}
    \item Additional information needed:
    \begin{enumerate}
        \item topographic data, data topografi pesisir dan daratan yang beresiko terkena tsunami
    \end{enumerate}
    \item Setting boundary conditions
    \item Adding land frictions
    \item Obtaining run-ups
\end{enumerate}
Hasil: ketinggian dan kecepatan gelombang pada setiap titik di daratan

\vspace{0.5cm}\hrule height 2pt\vspace{0.5cm}
\begin{center}
    \textbf{Damage Assessment}
\end{center}
Dalam tahap ini ketinggian dan kecepatan gelombang pada setiap koordinat darat telah diperoleh\\
\noindent\textbf{What we can obtain:}
\begin{enumerate}
    \item Energy:
    \[
        E = \frac{1}{2}\rho g h^2
    \]
    \begin{enumerate}
        \item $E$ energi gelombang
        \item $p$ densitas air
        \item $g$ gravitasi
        \item $h$ tinggi gelombang
    \end{enumerate}
    \item Impact Force on Structure:
    \[
        F = \rho g h A
    \]
    \begin{enumerate}
        \item $F$ gaya
        \item $\rho$ densitas air
        \item $g$ gravitasi
        \item $h$ kedalaman
        \item $A$ Area struktur
    \end{enumerate}
\end{enumerate}
Terapkan model yang mengevaluasi struktur bangunan bila diberikan gaya hidrodinamis.
Hasil: evaluasi struktur bangunan pada lokasi yang terkena pembanjiran

\vspace{0.5cm}\hrule height 2pt\vspace{0.5cm}

\noindent Additional Notes:
\begin{enumerate}
    \item hydrodynamic modelling, memodelkan seluruh proses dari tsunami generation sampai pembanjiran
    \item oceanografi, teknik sipil, dan matematika
\end{enumerate}

\vspace{0.5cm}\hrule height 2pt\vspace{0.5cm}

Referensi: 
Diskusi dengan ChatGPT

\end{document}
