\section{Closed Subset}
\begin{frame}{Closed Subset}
    \begin{tcolorbox}[enhanced,title=Definisi, frame style tile={width=\paperwidth}{\wallpaper}]
        Subset $A$ dari ruang topologi $(\nX, \nT)$ tertutup jika $\nX - A$ komplemennya terbuka.
    \end{tcolorbox}
\end{frame}

\begin{frame}{Contoh Closed Subset}
    \begin{tcolorbox}[enhanced,title=Contoh 38, frame style tile={width=\paperwidth}{\wallpaper}]
        Interval tertutup $\cic{a,b}, a < b$, di $\R$ tertutup terhadap \textit{usual topology}
        di $\R$ karena komplemennya, $\R-\cic{a,b}$, adalah $(-\infty, a) \cup (b,\infty)$.

        Jika $a \in \R$, maka $\{a\}$ tertutup terhadap \textit{usual topology} di $\R$ karena 
        $\R - \{a\}=\oio{-\infty,a} \cup \oio{a,\infty}$

        Pada \textit{finite complement topology} dari himpunan $\nX$, \textit{proper subset} dari
        $\nX$ tertutup jikka subset tersebut berhingga.

        Pada \textit{countable complement topology} dari himpunan $\nX$, \textit{proper subset} dari
        $\nX$ tertutup jikka subset tersebut terhitung.
    \end{tcolorbox}
\end{frame}

\begin{frame}{Contoh Closed Subset}
    \begin{tcolorbox}[enhanced,title=Contoh 38 (Penjelasan), frame style tile={width=\paperwidth}{\wallpaper}]
        Interval $[a,b],a<b$ di $\R$ tertutup terhadap \textit{usual topology}
        \begin{align*}
            &[a,b] \text{ tertutup}\\
            \iff &\R - [a,b] \text{ terbuka}\\
            \iff &(-\infty, a) \cup (b,\infty) \text{ terbuka}\\
            &(-\infty, a),(b,\infty)\in \nT\\
            \Longrightarrow& (-\infty, a) \cup (b,\infty) \in \nT\\
            \iff&(-\infty, a) \cup (b,\infty) \text{ terbuka}\\
        \end{align*}
        Untuk $a\in \R$ tertutup penjelasannya serupa.
    \end{tcolorbox}
\end{frame}

\begin{frame}{Contoh Closed Subset}
    \begin{tcolorbox}[enhanced,title=Contoh 38 (Penjelasan), frame style tile={width=\paperwidth}{\wallpaper}]
        Pada \textit{finite complement topology} dari himpunan $\nX$, \textit{proper subset} dari
        $\nX$ tutup jikka subset tersebut berhingga.
        \begin{align*}
            &A \subset \nX \text{ tertutup}\\
            \iff &\nX - A \text{ terbuka}\\
            \iff &\nX - A \in \nT\\
            \iff &\nX - (\nX - A) \text{ berhingga}\\
            \iff & A \text{ berhingga}
        \end{align*}
        Pada \textit{countable complement topology} dari himpunan $\nX$, \textit{proper subset} dari
        $\nX$ tutup jikka subset tersebut terhitung. (Penjelasan serupa)
    \end{tcolorbox}
\end{frame}


\begin{frame}{Contoh Closed Subset}
    \begin{tcolorbox}[enhanced,title=Contoh 39, frame style tile={width=\paperwidth}{\wallpaper}]
        Misalkan $(\nX, d)$ ruang metrik, misalkan $x \in \nX$, dan $\epsilon >0$. Maka
        $A = \{y \in \nX: d(x,y) \leq \epsilon\}$ adalah subset tertutup dari $\nX$.
    \end{tcolorbox}
    Definisi ini mirip dengan definisi Bola Terbuka (1.1) hanya $<$ diganti $\leq$.
    \begin{tcolorbox}[enhanced,title=Teorema 1.7 (1.2), frame style tile={width=\paperwidth}{\wallpaper}]
        Misalkan $(\nX, \nT)$ adalah ruang topologi. Misalkan $A$ subset $\nX$ dimana
        setiap $a \in A$, terdapat $U_a \in \nT$ sehingga $a \in U_a$ dan $U_a \subseteq A$.
        Maka $A \in \nT$.
    \end{tcolorbox}
\end{frame}

\begin{frame}{Contoh Closed Subset}
    \begin{tcolorbox}[enhanced,title=Contoh 39 (Analisa), frame style tile={width=\paperwidth}{\wallpaper}]
        $x$ tetap $\in \nX$, $\epsilon >0$ tetap.\\
        $A$ adalah semua titik yang "jaraknya" ke $x$ tidak lebih dari $\epsilon$.\\
        $\nX - A$ adalah semua titik yang "jaraknya" ke $x$ lebih dari $\epsilon$.\\
        $A$ tertutup $\iff$ $\nX-A$ terbuka $\iff$ $\nX-A \in \nT$.\\
        Gunakan Teorema 1.7 ($A_T = \nX - A$, $({\nX}_T,{\nT}_T) = (\nX ,\nT)$, $U_a =?$)\\
        Ambil sembarang $y\in \nX-A$ dan misalkan $r_y = d(x,y)$.\\
        Maka $r_y > \epsilon \iff r_y - \epsilon > 0$, misalkan $\delta_y=r_y-\epsilon$.
    \end{tcolorbox}
\end{frame}

\begin{frame}{Contoh Closed Subset}
    \begin{tcolorbox}[enhanced,title=Contoh 39 (Analisa), frame style tile={width=\paperwidth}{\wallpaper}]
        Klaim: $U_a = B(y,\delta_y) \in \nT$.\\
        Jelas $y \in B(y, \delta_y)$. Adib $B(y,\delta_y) \subseteq \nX-A$.
        \begin{enumerate}
            \item Ambil sembarang $z \in B(y, \delta_y)$ maka $d(y,z) < \delta_y$.\\
            \item Dari sifat metrik $d(x,z)+d(z,y) \geq d(x,y) \iff d(x,z) \geq d(x,y)-d(z,y)$\\
            \item Dari 1, $d(y,z) < \delta y \iff -d(y,z) > -\delta_y$\\$\iff d(y,z) > -(r_y-\epsilon)$
            \item Dari 1 dan 2, $d(x,z) \geq d(x,y)-d(z,y) > d(x,y)-\delta_y$\\
            $\iff d(x,z) > d(x,y) - (r_y-\epsilon)$\\$\iff d(x,z) > r_y - (r_y-\epsilon) \iff d(x,z) > \epsilon$.
            \item $z \notin A$ sehingga $z \in \nX - A$, maka $B(y,\delta_y) \subseteq \nX-A$.
        \end{enumerate}
        
    \end{tcolorbox}
\end{frame}

\begin{frame}{Teorema Closed Subset}
    \begin{tcolorbox}[enhanced,title=Teorema 1.18, frame style tile={width=\paperwidth}{\wallpaper}]
        Misalkan $(\nX,\nT)$ adalah ruang topologi. Maka kondisi berikut berlaku:
        \begin{enumerate}
            \item $\nX$ dan $\emptyset  $ adalah subset tertutup.
            \item Jika $\nA$ adalah \textit{finite family of closed subset} dari $\nX$, maka
            $\bigcup\{A:A\in\nA\}$ adalah himpunan tertutup.
            \item Jika $\nA$ adalah \textit{family of closed subsets} dari $\nX$, maka
            $\bigcap\{A:A\in\nA\}$ adalah himpunan tertutup.
        \end{enumerate}
    \end{tcolorbox}
\end{frame}

\begin{frame}{Teorema Closed Subset}
    \begin{tcolorbox}[enhanced,title=Teorema 1.18 (Bukti), frame style tile={width=\paperwidth}{\wallpaper}]
        $(\nX,\nT)$ adalah ruang topologi
        \begin{enumerate}
            \item $\nX$ tertutup karena $\nX-\nX = \emptyset \in \nT$, terbuka.\\
            $\emptyset$ tertutup karena $\nX-\emptyset = \nX \in \nT$, terbuka.
            \item Diberikan ${A_1,A_2,\dots,A_n}=\nA$ dengan $A_i$ tertutup, $i=1,2,\dots,n$.
            Adib $\bigcup\{A:A\in\nA\}$ tertutup.
        \end{enumerate}
    \end{tcolorbox}
\end{frame}

\begin{frame}{Teorema Closed Subset}
    \begin{tcolorbox}[enhanced,title=Teorema 1.18 (Bukti), frame style tile={width=\paperwidth}{\wallpaper}]
        Karena $A_1,\dots,A_n$ tertutup, maka $\nX-A_1,\dots,\nX-A_n$ terbuka.  
            \begin{align*}
                &\nX-A_1,\dots,\nX-A_n \in \nT\\
                &\text{dengan sifat 2 ruang topologi}\\
                \Longrightarrow&\nX-A_1\cap\dots\cap\nX-A_n \in \nT\\
                \iff&(\nX\cap \comp{A_1})\cap\dots\cap(\nX\cap \comp{A_n})\in \nT\\
                \iff&\nX\cap \comp{A_1}\cap \comp{A_2}\cap\dots\cap \comp{A_n}\in \nT\\
                \iff&\nX-\comp{\oio*{\comp{A_1}\cap \comp{A_2}\cap\dots\cap \comp{A_n}}}\in\nT\\
                \iff&\nX-\oio*{A_1\cup A_2\cup\dots\cup A_n}\in\nT\\
                \iff&\nX-\bigcup\{A:A\in\nA\}\in\nT
            \end{align*}
            Maka $\bigcup\{A:A\in\nA\}$ tertutup.
    \end{tcolorbox}
\end{frame}

\begin{frame}{Teorema Closed Subset}
    \begin{tcolorbox}[enhanced,title=Teorema 1.18 (Bukti), frame style tile={width=\paperwidth}{\wallpaper}]
        \begin{enumerate}
            \setcounter{enumi}{2}
            \item Diberikan $\forall A\in\nA$ dengan $A$ tertutup.
            Adib $\bigcap\{A:A\in\nA\}$ tertutup.\\
            Karena $A$ tertutup $\forall A \in \nA$, maka $\nX-A$ terbuka $\forall A\in \nA$.
        \end{enumerate}
    \end{tcolorbox}
\end{frame}

\begin{frame}{Teorema Closed Subset}
    \begin{tcolorbox}[enhanced,title=Teorema 1.18 (Bukti), frame style tile={width=\paperwidth}{\wallpaper}]
        Karena $A$ tertutup $\forall A \in \nA$, maka $\nX-A$ terbuka $\forall A\in \nA$.
        \begin{align*}
            &\nX-A \in \nT, \forall A\in \nA\\
            &\text{sifat 3 ruang topologi}\\
            \Longrightarrow& \bigcup\{\nX-A: A\in \nA\}\in \nT\\
            \iff &\bigcup\{\nX\cap \comp{A}:A\in\nA\}\in \nT\\
            \iff &\nX\cap\bigcup\{\comp{A}:A\in\nA\}\in \nT\\
            \iff &\nX-\comp{\bigcup\{\comp{A}:A\in\nA\}}\in \nT\\
            \iff &\nX-\bigcap\{A:A\in\nA\}\in \nT
        \end{align*}
        Maka $\bigcap\{A:A\in\nA\}$ tertutup.
    \end{tcolorbox}
\end{frame}

\begin{frame}{Contoh Subset yang Tidak Terbuka Maupun Tertutup}
    \begin{tcolorbox}[enhanced,title=Contoh 40(a), frame style tile={width=\paperwidth}{\wallpaper}]
        Terhadap \textit{usual topology} di $\R$, jika $a,b\in\R$, dan $a<b$,
        maka $\cio{a,b}$ bukan subset terbuka maupun tertutup.
    \end{tcolorbox}
    Analisa:\\
    Jelas $\cio{a,b}$ tidak terbuka karena tidak terdapat $c,d\in\R$ sehingga
    $\cio{a,b} = \oio{c,d}$.\\
    Lalu $\cio{a,b}$ tidak tertutup karena $\R-\cio{a,b} = \oio{-\infty,a}\cup\cio{b,\infty}$ 
    tidak terbuka.
\end{frame}

\begin{frame}{Contoh Subset yang Terbuka Juga Tertutup}
    \begin{tcolorbox}[enhanced,title=Contoh 40(b), frame style tile={width=\paperwidth}{\wallpaper}]
        Terhadap \textit{lower-limit topology} di $\R$, jika $a,b\in\R$, dan $a<b$,
        maka $\cio{a,b}$ subset terbuka juga tertutup.
    \end{tcolorbox}
    Analisa:\\
    Setiap himpunan buka untuk \textit{lower-limit topology} di $R$ berbentuk $\cio{x,y}$
    dengan $x < y$ dan $x,y\in R$. Maka $\cio{a,b}\in \nT$ dan $\cio{a,b}$ terbuka.\\
    Lalu $\cio{a,b}$ tertutup karena $\R-\cio{a,b} = \oio{-\infty,a}\cup\cio{b,\infty}\in\nT$ 
    sehingga $\R-\cio{a,b}$ terbuka.
\end{frame}

\section{Closure}
\begin{frame}{Closure}
    \begin{tcolorbox}[enhanced,title=Definisi, frame style tile={width=\paperwidth}{\wallpaper}]
        \textit{Closure} $\comp{A}$ dari subset $A$ dari ruang topologi $(\nX,\nT)$, 
        adalah irisan \textbf{himpunan tertutup yang mengandung A}.
    \end{tcolorbox}
Catatan:\\
Ini berarti closure $\comp{A}$ adalah himpunan tertutup "terkecil" yang mengandung $A$. 
Sehingga untuk sembarang $B$ dengan $A\subseteq B$, berlaku juga $\comp{A} \subseteq B$.\\
Berdasarkan teorema 1.18, $\comp{A}$ adalah himpunan tertutup juga.
\end{frame}

\begin{frame}{Teorema Closure}
Bagaimana cara menentukan suatu elemen di $\nX$ juga elemen di $\comp{A}$?  
\begin{tcolorbox}[enhanced,title=Teorema 1.19, frame style tile={width=\paperwidth}{\wallpaper}]
    Misalkan $A$ subset dari ruang topologi $(\nX,\nT)$, dan $x\in \nX$.
    Maka $x\in\comp{A}$ jikka setiap lingkungan dari $x$ tidak kosong ketika diiriskan dengan $A$.
\end{tcolorbox}
\end{frame}

\begin{frame}{Teorema Closure} 
    \begin{tcolorbox}[enhanced,title=Teorema 1.19 (Bukti), frame style tile={width=\paperwidth}{\wallpaper}]
        $(\Rightarrow)$ Asumsikan $x\notin \comp{A}$ dan irisan sembarang lingkungan dari $x$ dan $A$ tak kosong.\\
        \begin{itemize}
            \item Karena $x\notin \comp{A}$, maka terdapat $C=\comp{A}$, sehingga $A\subseteq C$ dan $x\notin C$. 
            \item Maka $\nX-C$ terbuka dan $x\in \nX-C$.
            \item Sehingga $\nX-C$ adalah lingkungan dari $x$ dengan $\nX-C \cap A = \emptyset$ (ingat $A\subseteq C$).
        \end{itemize}
        
        Ini berkontradiksi dengan asumsi.
    \end{tcolorbox}
\end{frame}

\begin{frame}{Teorema Closure} 
    \begin{tcolorbox}[enhanced,title=Teorema 1.19 (Bukti), frame style tile={width=\paperwidth}{\wallpaper}]
        $(\Leftarrow)$ Asumsikan $x\in \comp{A}$ dan terdapat irisan lingkungan dari $x$ dan $A$ yang kosong.
        \begin{itemize}
            \item Misalkan irisan lingkungan $x$ dengan $A$ yang kosong tersebut adalah $U$.
            \item Karena $U$ lingkungan dari $x$ maka $x \in U$ dan $U$ terbuka.
            \item $U$ terbuka maka $\nX-U$ tertutup.
            \item Karena $U \cap A = \emptyset$ maka $A\subseteq \nX-U$.
            \item $\nX-U$ himpunan tertutup yang mengandung $A$, maka $\comp{A}\subseteq \nX-U$.
            \item $x \in U$, maka $x \notin \nX-U$ juga $x\notin \comp{A}$
        \end{itemize}
        Ini berkontradiksi dengan asumsi
    \end{tcolorbox}
\end{frame}

\begin{frame}{Limit Point}
    \begin{tcolorbox}[enhanced,title=Definisi, frame style tile={width=\paperwidth}{\wallpaper}]
        Misalkan $A$ subset dari ruang topologi $(\nX,\nT)$. Titik $x$ di $\nX$ adalah
        \textbf{limit point} dari $A$ jika setiap lingkungan $x$ mengandung suatu titik di $A$ yang bukan $x$.\\
        $A'$ menyatakan himpunan semua limit point dari $A$.
    \end{tcolorbox}
Catatan:\\
Jika $x$ limit point dari $A$ dan $U$ sembarang lingkungan dari $x$, maka
$U \cap (A-\{x\})\neq\emptyset$.
\end{frame}

\begin{frame}{Contoh Limit Point}
    \begin{tcolorbox}[enhanced,title=Contooh 41 (a), frame style tile={width=\paperwidth}{\wallpaper}]
        Terhadap \textit{usual topology} di $\R$, jika $a,b\in\R$ dan $a<b$,
        maka $a$ adalah limit point terhadap $\cic{a,b}$, $\cio{a,b}$, dan $\oio{a,b}$.
    \end{tcolorbox}
Analisa: \\
Terhadap \textit{usual topology} di $\R$ lingkungan dari $a$ adalah 
$\oio{a-\epsilon,a+\delta}$ dengan $\epsilon,\delta > 0$\\
Irisan sembarang lingkungan dari $a$ terhadap $\cic{a,b}$, $\cio{a,b}$,$\oio{a,b}$:
\begin{itemize}
    \item $\oio{a-\epsilon,a+\delta}\cap\cic{a,b} = \cic{a,b}$ atau $\cio{a,a+\delta}$
    \item $\oio{a-\epsilon,a+\delta}\cap\cio{a,b} = \cio{a,\min\{b,\delta\}}$
    \item $\oio{a-\epsilon,a+\delta}\cap\oio{a,b} = \oio{a,\min\{b,\delta\}}$
\end{itemize}
Karena $a<b$ dan $a<a+\delta$, setiap kemungkinan ini tidak mungkin himpunan kosong.
\end{frame}

\begin{frame}{Contoh Bukan Limit Point}
    \begin{tcolorbox}[enhanced,title=Contooh 41 (b), frame style tile={width=\paperwidth}{\wallpaper}]
        Terhadap \textit{lower-limit topology} di $\R$, jika $a,b\in\R$ dan $a<b$,
        maka $b$ bukan limit point terhadap $\cio{a,b}$.
    \end{tcolorbox}
Analisa: \\
Counter-example, pilih himpunan $\cio{b,b+\delta}$ dengan $\delta > 0$. \\
Maka $\cio{b,b+\delta}$ lingkungan dari $b$ karena $b\in\cio{b,b+\delta}$ 
dan $\cio{b,b+\delta}\in \nT$.\\
Tetapi $\cio{a,b}\cap\cio{b,b+\delta}=\emptyset$.\\
$b$ bukan limit point terhadap $\cio{a,b}$.
\end{frame}

\begin{frame}{Limit Point Secara Intuitif}
    \begin{tcolorbox}[enhanced,title=Contooh 42, frame style tile={width=\paperwidth}{\wallpaper}]
        Misalkan $(\nX,d)$ ruang metrik, misalkan $A\subseteq \nX$, dan $x\in\nX$.
        Maka $x$ adalah limit point dari $A$ jikka $\forall e>0, \exists y\in\nX$ 
        sehingga $y\neq x$ dan $d(x,y)<\epsilon$.
    \end{tcolorbox}
Notes:\\
Secara intuitif, $x$ limit point dari $A$ jikka terdapat elemen-elemen di $A$ yang 
bukan $x$ dan "dekat" dengan $x$.
\end{frame}
