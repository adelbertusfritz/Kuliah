% ===== Setup Page Layout =====
\documentclass{article}
\usepackage{geometry}
 \geometry{
 a4paper,
 total={6in, 8in},
 }
\usepackage{graphicx}
% ===== Setup Font =====
\usepackage[sfdefault,lf]{carlito}
\usepackage[T1]{fontenc}
\renewcommand*\oldstylenums[1]{\carlitoOsF #1}

% ==== Import Math Packages =====
\usepackage{amsmath, amssymb, amsthm}
\usepackage{mathtools}

% ==== Import Styling Packages =====
\usepackage{enumitem}
\usepackage[pages=some, placement=bottom]{background}
\usepackage{moresize}
\usepackage{relsize}
\usepackage{hyperref}
\hypersetup{colorlinks=true,allcolors=blue}
\usepackage{hypcap}
\usepackage{verbatim}
\usepackage[normalem]{ulem}

\usepackage{hyperref}

% ==== Custom Declarations =====
\DeclarePairedDelimiter\abs{\lvert}{\rvert}
\DeclarePairedDelimiter\floor{\lfloor}{\rfloor}
\DeclarePairedDelimiter\cic{[ }{] }
\DeclarePairedDelimiter\oic{( }{] }
\DeclarePairedDelimiter\cio{[ }{) }
\DeclarePairedDelimiter\oio{( }{) }
\DeclarePairedDelimiter\set{\{ }{\} }
\DeclarePairedDelimiter\brk{(}{)}
\newcommand{\drv}[2]{\frac{d}{d#1}\brk*{#2}}
\newcommand{\drvL}[2]{D_{#1}\brk*{#2}}
\newcommand{\ds}{\displaystyle}
\newcommand{\eval}[3]{\left.\brk*{#1}\right\rvert_{#2}^{#3}}
\newcommand{\R}{\mathbb{R}}
\newcommand{\vecu}{\textbf{u}}

\title{Tugas 0}
\author{Fritz Adelbertus Sitindaon}
\date{}

\begin{document}
\begin{flushright}
    \section*{Tugas 0: Pemrograman Fungsional}
\end{flushright}

Judul: \href{{https://www.youtube.com/watch?v=E8I19uA-wGY&list=RDQMz2xSS6pH_wo&start_radio=1}}
{Functional Programming Design Patterns by Scott Wlaschin}\\

Pemrograman fungsional (FP) adalah paradigma pemrograman yang menekankan 
evaluasi fungsi matematis dan menghindari perubahan status dan data yang 
dapat diubah. Pemrograman fungsional fokus pada prinsip fungsi murni dan 
immutabilitas yang menjadikannya pendekatan baik untuk menulis 
\textit{clean  code}. Beberapa poin yang ingin disampaikan sebagai berikut:

\begin{enumerate}
    \item \textbf{Fokus pada Input dan Output}

    Inti dari pemrograman fungsional adalah penekanan pada input dan output. 
    Dalam FP, fungsi dirancang untuk menerima input dan menghasilkan 
    output tanpa efek samping. Ini berarti bahwa dengan input yang 
    sama, sebuah fungsi akan selalu menghasilkan output yang sama, membuat 
    perilaku program lebih dapat diprediksi dan lebih mudah untuk \textit{debug}. 
    Ketiadaan efek samping memastikan bahwa status sistem tetap konsisten.
    \item \textbf{Semuanya adalah Fungsi}

    Dalam pemrograman fungsional, segala sesuatu diperlakukan sebagai fungsi. 
    Sementara pemrograman berorientasi objek (OOP) dibangun di sekitar konsep 
    objek dan metode, FP menyederhanakan ini dengan memperlakukan objek 
    sebagai fungsi. Alih-alih memiliki objek dengan metode yang mengubah 
    status, FP mendorong penggunaan fungsi murni yang menerima input dan 
    mengembalikan output tanpa mengubah status eksternal apa pun.

    \item \textbf{Fungsi dibuat Satu Parameter}

    Prinsip utama dalam pemrograman fungsional adalah bahwa setiap fungsi 
    sebaiknya menerima hanya satu parameter. Sebuah fungsi yang 
    menerima tiga parameter dapat diubah menjadi serangkaian tiga fungsi, 
    masing-masing menerima satu parameter. Pendekatan ini tidak hanya 
    menyederhanakan komposisi fungsi tetapi juga membuat kode lebih modular 
    dan dapat digunakan kembali.

    \item \textbf{Monad: Mencegah Pyramid of Doom}

    Monad adalah konsep fundamental dalam pemrograman fungsional yang 
    membantu mengelola efek samping dan mengontrol alur. Salah satu masalah 
    umum dalam pemrograman adalah "pyramid of doom," di mana callback 
    bersarang menyebabkan kode yang dalam dan sulit dibaca. Monad 
    menyelesaikan masalah ini dengan memungkinkan pengembang untuk merangkai 
    operasi dalam cara yang linear dan dapat dibaca.

    \item \textbf{Option: Meningkatkan Keterbacaan dan Dokumentasi}

    Dalam pemrograman fungsional, konsep \textit{option} dipakai
    untuk merepresentasikan nilai yang mungkin ada atau tidak ada. Daripada 
    menggunakan nilai null, yang dapat menyebabkan kesalahan runtime jika 
    tidak ditangani dengan baik, \textit{option} menyediakan cara aman untuk 
    menangani nilai opsional. Ini tidak hanya mencegah \textit{bug} umum 
    tetapi juga membuat dokumentasi lebih mudah dibaca. Dengan menunjukkan 
    dengan jelas nilai mana yang opsional, \textit{option} meningkatkan 
    kejelasan kode dan memudahkan pengembang lain untuk memahami maksud di 
    baliknya.

    \item \textbf{Monoid: Komposabilitas dan Kesederhanaan}

    Dalam pemrograman fungsional, monoid digunakan untuk memodelkan suatu 
    operasi yang dapat digabungkan bersama. Sifat kunci dari monoid adalah 
    bahwa kombinasi dari monoid juga merupakan monoid, yang memungkinkan 
    komposisi fungsi dan operasi yang mudah. Ini menyederhanakan proses 
    pengembangan dan menghasilkan kode yang lebih modular. Monoid sangat 
    berguna dalam skenario di mana beberapa operasi perlu digabungkan.
\end{enumerate}
\end{document}
