% ===== Setup Page Layout =====
\documentclass{article}
\usepackage{geometry}
 \geometry{
 a4paper,
 total={15cm, 20cm},
 }
\usepackage{graphicx}
% ===== Setup Font =====
\usepackage[sfdefault,lf]{carlito}
\usepackage[T1]{fontenc}
\renewcommand*\oldstylenums[1]{\carlitoOsF #1}

% ==== Import Math Packages =====
\usepackage{amsmath, amssymb, amsthm}
\usepackage{mathtools}

\newtheorem{theorem}{Teorema}[section]
\newtheorem{corollary}{Akibat}[theorem]
\newtheorem{lemma}[theorem]{Lemma}
\newtheorem{definition}[theorem]{Definisi}

% ==== Import Styling Packages =====
\usepackage{enumitem}
\usepackage[pages=some, placement=bottom]{background}
\usepackage{moresize}
\usepackage{relsize}
\usepackage{hyperref}
\hypersetup{colorlinks=true,allcolors=blue}
\usepackage{hypcap}
\usepackage{verbatim}
\usepackage[normalem]{ulem}

\usepackage{hyperref}

% ==== Custom Declarations =====
\DeclarePairedDelimiter\abs{\lvert}{\rvert}
\DeclarePairedDelimiter\floor{\lfloor}{\rfloor}
\DeclarePairedDelimiter\cic{[ }{] }
\DeclarePairedDelimiter\oic{( }{] }
\DeclarePairedDelimiter\cio{[ }{) }
\DeclarePairedDelimiter\oio{( }{) }
\DeclarePairedDelimiter\set{\{ }{\} }
\DeclarePairedDelimiter\brk{(}{)}
\DeclarePairedDelimiter\vct{\langle}{\rangle}
\newcommand{\Mod}[1]{\ (\mathrm{mod}\ #1)}
\newcommand{\drv}[2]{\frac{d}{d#1}\brk*{#2}}
\newcommand{\drvL}[2]{D_{#1}\brk*{#2}}
\newcommand{\ds}{\displaystyle}
\newcommand{\eval}[3]{\left.\brk*{#1}\right\rvert_{#2}^{#3}}
\newcommand{\R}{\mathbb{R}}
\newcommand{\Rubik}{
    \draw[black, thick] (0,0) -- (2.598,1.5);
    \draw[black, thick] (0,0) -- (-2.598,1.5);
    \draw[black, thick] (0,0) -- (0,-3);
    \draw[black, thick] (0,-3) -- (2.598,-1.5);
    \draw[black, thick] (0,-3) -- (-2.598,-1.5);
    \draw[black, thick] (0,-2) -- (2.598,-0.5);
    \draw[black, thick] (0,-2) -- (-2.598,-0.5);
    \draw[black, thick] (0,-1) -- (2.598,0.5);
    \draw[black, thick] (0,-1) -- (-2.598,0.5);
    \draw[black, thick] (2.598,-1.5) -- (2.598,1.5);
    \draw[black, thick] (-2.598,-1.5) -- (-2.598,1.5);
    \draw[black, thick] (0,3) -- (2.598,1.5);
    \draw[black, thick] (0,3) -- (-2.598,1.5);
    \draw[black, thick] (0.867,0.5) -- (0.867,-2.5);
    \draw[black, thick] (1.732,1) -- (1.732,-2);
    \draw[black, thick] (-0.867,0.5) -- (-0.867,-2.5);
    \draw[black, thick] (-1.732,1) -- (-1.732,-2);
    \draw[black, thick] (-0.867,0.5) -- (1.732,2);
    \draw[black, thick] (-1.732,1) -- (0.867,2.5);
    \draw[black, thick] (0.867,0.5) -- (-1.732,2);
    \draw[black, thick] (1.732,1) -- (-0.867,2.5);
}

\newcommand{\RubikU}{
    \draw[black, thick] (0,0) -- (2.598,1.5);
    \draw[black, thick] (0,0) -- (-2.598,1.5);
    \draw[black, thick] (0,3) -- (0,6);
    \draw[black, thick] (0,6) -- (2.598,4.5);
    \draw[black, thick] (0,6) -- (-2.598,4.5);
    \draw[black, thick] (0,4) -- (2.598,2.5);
    \draw[black, thick] (0,4) -- (-2.598,2.5);
    \draw[black, thick] (0,5) -- (2.598,3.5);
    \draw[black, thick] (0,5) -- (-2.598,3.5);
    \draw[black, thick] (2.598,4.5) -- (2.598,1.5);
    \draw[black, thick] (-2.598,4.5) -- (-2.598,1.5);
    \draw[black, thick] (0,3) -- (2.598,1.5);
    \draw[black, thick] (0,3) -- (-2.598,1.5);
    \draw[black, thick] (0.867,5.5) -- (0.867,2.5);
    \draw[black, thick] (1.732,5) -- (1.732,2);
    \draw[black, thick] (-0.867,2.5) -- (-0.867,5.5);
    \draw[black, thick] (-1.732,2) -- (-1.732,5);
    \draw[black, thick] (-0.867,0.5) -- (1.732,2);
    \draw[black, thick] (-1.732,1) -- (0.867,2.5);
    \draw[black, thick] (0.867,0.5) -- (-1.732,2);
    \draw[black, thick] (1.732,1) -- (-0.867,2.5);
}

\title{Responsi 4}
\author{Fritz Adelbertus Sitindaon}
\date{}

\begin{document}
\begin{flushright}
    \section*{Responsi 4 Analisis 1}
    \textbf{Tim Asisten Dosen}
\end{flushright}


\vspace{0.5cm}\hrule height 2pt\vspace{0.5cm}



\section{Konfigurasi Rubik}

Konfigurasi rubik akan dinyatakan dalam suatu notasi Grup sehingga dapat dihubungkan
dengan Grup Rubik $\vct{F,L,U,D,R,B}$. Konfigurasi rubik dapat disusun dari 
5 grup.
\begin{enumerate}
    \item Orientasi sudut rubik ($C_3$), terdiri dari 3 permutasi yaitu $C_3=\set{e,(123),(132)}$. \begin{center}
    \begin{tikzpicture}[scale=0.75]
        \Rubik
        \draw[<-, thick, blue] (0.7,0) arc(0:300:0.7);
        \node at (0,0.35) {$x_1$};
        \node at (0.35,-0.25) {$x_2$};
        \node at (-0.35,-0.25) {$x_3$};
    \end{tikzpicture}
\end{center}
    \item Orientasi sisi rubik ($C_2$), terdiri dari 2 permutasi yaitu $C_2=\set{e,(12)}$.\begin{center}
    \begin{tikzpicture}[scale=0.75]
        \Rubik
        \draw[blue, thick] (1.156,0.63) -- (1.156,0.3);
        \draw[->, blue, thick] (1.156,0.63) -- (0.75,0.9);
        \draw[blue, thick] (1.444,0.77) -- (1.1,1);
        \draw[->, blue, thick] (1.444,0.77) -- (1.444,0.3);
        \node at (0.8,1.1) {$y_1$};
        \node at (1.3,0.1) {$y_2$};
    \end{tikzpicture}
\end{center}
    \item Orientasi tengah ($C_4$), terdiri dari 4 permutasi yaitu $C_4=\set{e,(1234),(1432),(13)(24)}$ \begin{center}
    \begin{tikzpicture}[scale=0.75]
        \Rubik
        \draw[<-, thick, blue] (1.7,-0.75) arc(0:300:0.4);
        \draw[->, black, thick] (1.3,-1) -- (1.3,-0.5);
        \node at (1.3,0.1) {$z_1$};
        \node at (2.1,-0.4) {$z_2$};
        \node at (0.5,-1.2) {$z_3$};
        \node at (1.3,-1.6) {$z_4$};
    \end{tikzpicture}
\end{center}
    \item Posisi sudut rubik ($S_8$), adalah grup simetri 8 \begin{center}
    \begin{tikzpicture}[scale=0.75]
        \Rubik
        \node at (0,0.5) {$a_1$};
        \node at (-1.732,1.5) {$a_2$};
        \node at (1.732,1.5) {$a_3$};
        \node at (0,2.5) {$a_4$};
        \node at (-2.5,-2) {$a_6$};
        \node at (2.5,-2) {$a_7$};
        \node at (0,-3.3) {$a_5$};
    \end{tikzpicture}
\end{center}
    \item Posisi sisi rubik ($S_{12}$), adalah grup simetri 12 \begin{center}
    \begin{tikzpicture}[scale=0.75]
        \Rubik
        \node at (0.9,1) {$b_1$};
        \node at (-0.9,1) {$b_2$};
        \node at (-0.9,2) {$b_3$};
        \node at (0.9,2) {$b_4$};
        \node at (3,-0.1) {$b_7$};
        \node at (1.3,-2.6) {$b_9$};
        \node at (-2.1,-0.4) {$b_5$};
        \node at (-0.5,-1.2) {$b_8$};
        \node at (-1.3,-2.6) {$b_{10}$};
    \end{tikzpicture}
\end{center}
\end{enumerate}
Menggabungkan 5 grup ini akan menghasilkan semua kemungkinan konfigurasi rubik.

\subsection{Direct Product}

\begin{definition}
    Misalkan $G_1, G_2$ adalah grup. \textbf{Direct product} dari $G_1$ dan $G_2$ adalah
    himpunan $G_1\times G_2$ dengan operasi $(g_1,g_2)\cdot(g'_1,g'_2) = (g_1g'_1,g_2g'_2)$ dengan $g_1,g'_1\in G_1$ 
    dan $g_2,g'_2\in G_2$.
\end{definition}

\noindent Rubik memiliki 8 sudut dan masing-masing memiliki orientasi sudut yang saling bebas. 8 
grup $C_3$ ini digabung dengan \textbf{direct product}, untuk menghasilkan 
$\underbrace{C_3\times C_3\dots\times C_3}_\text{8 kali}= C_3^8$.\\
Dengan cara yang serupa, memerhatikan orientasi sisi dan tengah akan menghasilkan $C_2^{12}$ dan $C_4^6$ 
secara berturut-turut. 

Telah digabungkan semua kemungkinan orientasi pada masing-masing komponen rubik (sudut, sisi, tengah).
Sekarang akan digabungkan orientasi tersebut dengan posisi.

\subsection{Wreath Product}

% \begin{definition}
%     Misalkan $G_1\times G_2$ adalah subgrup. Maka $A=G_1\rtimes G_2$ adalah \textbf{semi-direct product}
%     jika
%     \begin{enumerate}
%         \item $A=G_1G_2$
%         \item $G_1\cap G_2= e_A
%     \end{enumerate}
% \end{definition}

\begin{definition}
    Misalkan $X$ himpunan berhingga, $G$ grup dan $H$ grup yang \textbf{beraksi} pada $X$. Tetapkan 
    label untuk $X$ seperti $\set{x_1,x_2,\dots,x_t}$, dengan $|X|=t$. Lalu misalkan $G^t$ 
    adalah \textbf{direct product} dari $G$ terhadap dirinya sebanyak $t$ kali. $G^t\wr H = G \rtimes H$
    adalah \textbf{wreath product} dengan $H$ beraksi pada $G^t$ menurut aksinya terhadap $X$.
\end{definition}

Wreath product dari direct product dengan suatu grup simetri $S_n$ akan membentuk semua kemungkinan 
posisi dari direct product. Dengan wreath product, kemungkinan orientasi sudut dan posisi sudut dapat digabung 
dengan notasi $C_3^8\wr S_8$. Begitu juga dengan orientasi dan posisi dari sisi digabung dengan notasi 
$C_2^{12}\wr S_{12}$. Karena bagian tengah tidak berpindah posisi, maka akan tetap berbentuk $C_4^6$. 
Terakhir karena konfigurasi rubik pada bagian tengah, sisi dan sudut saling bebas maka dapat 
digabung dengan \textbf{direct product}

\begin{lemma}
    $C_3^8\wr S_8 \times C_2^{12}\wr S_{12}\times C_4^6$ membentuk grup konfigurasi rubik.
\end{lemma}

\section{Sistem Penomoran}
Perhatikan penetapan sistem penomoran pada rubik ini.
\begin{center}
    \begin{tikzpicture}
        \Rubik
        \node at (0,0.5) {$0$};
        \node at (-1.732,1.5) {$0$};
        \node at (1.732,1.5) {$0$};
        \node at (0,2.5) {$0$};
        \node at (0.9,1) {$0$};
        \node at (-0.9,1) {$0$};
        \node at (-0.9,2) {$0$};
        \node at (0.9,2) {$0$};
        \node at (-2.15,-0.3) {$0$};
        \node at (-2.15,0.7) {$2$};
        \node at (-2.15,-1.3) {$1$};
        \node at (-0.45,-0.3) {$2$};
        \node at (-0.45,-1.3) {$0$};
        \node at (-0.45,-2.3) {$1$};
        \node at (-1.3,0.3) {$1$};
        \node at (-1.3,-1.7) {$1$};
        \node at (2.15,-0.3) {$1$};
        \node at (2.15,0.7) {$2$};
        \node at (2.15,-1.3) {$1$};
        \node at (0.45,-0.3) {$1$};
        \node at (0.45,-1.3) {$1$};
        \node at (0.45,-2.3) {$2$};
        \node at (1.3,0.3) {$1$};
        \node at (1.3,-1.7) {$1$};
        \node at (1.15,-0.7) {$0$};
        \node at (-1.45,-0.7) {$0$};
        \node at (-0.2,1.5) {$0$};
        \node [rotate=90] at (1.45,-0.7) {$1$};
        \node [rotate=90] at (-1.15,-0.7) {$1$};
        \node [rotate=90] at (0.2,1.5) {$1$};
        \node at (0,1.2) {U};
        \node at (1.15,-1.1) {F};
        \node at (-1.15,-1.1) {L};
        
    \end{tikzpicture}
    \begin{tikzpicture}
        \RubikU
        \node at (0,0.5) {$0$};
        \node at (-1.732,1.5) {$0$};
        \node at (1.732,1.5) {$0$};
        \node at (0,2.5) {$0$};
        \node at (0.9,1) {$0$};
        \node at (-0.9,1) {$0$};
        \node at (-0.9,2) {$0$};
        \node at (0.9,2) {$0$};
        \node at (-2.15,4.2) {$1$};
        \node at (-2.15,3.2) {$0$};
        \node at (-2.15,2.2) {$2$};
        \node at (-0.45,5.2) {$2$};
        \node at (-0.45,4.2) {$0$};
        \node at (-0.45,3.2) {$1$};
        \node at (-1.3,4.7) {$1$};
        \node at (-1.3,2.7) {$1$};
        \node at (2.15,4.2) {$2$};
        \node at (2.15,3.2) {$1$};
        \node at (2.15,2.2) {$1$};
        \node at (0.45,5.2) {$1$};
        \node at (0.45,4.2) {$1$};
        \node at (0.45,3.2) {$2$};
        \node at (1.3,4.7) {$1$};
        \node at (1.3,2.7) {$1$};
        \node at (1.15,3.7) {$0$};
        \node at (-1.45,3.7) {$0$};
        \node at (-0.2,1.5) {$0$};
        \node [rotate=90] at (1.45,3.7) {$1$};
        \node [rotate=90] at (-1.15,3.7) {$1$};
        \node [rotate=90] at (0.2,1.5) {$1$};
        \node at (1.15,4.1) {B};
        \node at (-1.15,4.1) {R};
        \node at (0,1.2) {D};
    \end{tikzpicture}
\end{center}
Bilangan orientasi dari suatu sudut, sisi, tengah pada suatu konfigurasi rubik adalah selisih 
modular terhadap konfigurasi awal seperti pada gambar di atas.

\begin{lemma}
    Setiap gerakan $m \in \set{F,L,R,B}$, bilangan orientasi pada sudut yang berputar akan 
    berubah sebanyak $1\Mod 3$ atau $2\Mod 3$, dan $m\in \set{U,D}$ bilangan orientasi tidak berubah $(0\Mod 3)$.
\end{lemma}

\begin{lemma}
    Setiap gerakan $m \in \set{F,L,U,D,R,B}$, bilangan orientasi pada sisi yang berputar akan 
    berubah sebanyak $1\Mod 2$.
\end{lemma}

\begin{lemma}
    Setiap gerakan $m \in \set{F,L,U,D,R,B}$, bilangan orientasi pada tengah yang berputar akan 
    berubah sebanyak $0\Mod 2$ atau $1\Mod 2$.
\end{lemma}

Pada penjelasan selanjutnya, setiap pengambilan elemen dari $C_3^8,\,C_2^{12}$ atau $C_4^6$ akan dipetakan
ke bilangan orientasi masing-masing.

\section{Teorema Fundamental Rubik Pertama}
Permasalahan bisa tidaknya suatu konfigurasi rubik diselesaikan menghasilkan teorema 
pertama pada grup Rubik yang menunjukan syarat cukup untuk menentukan \textit{solveability} dari suatu konfigurasi rubik.
\begin{theorem}
    Misalkan $u \in C_4^6,\,v\in C_3^8,\,w\in C_2^{12},\,r\in S_8,\,s\in S_{12}$. $(u,v,w,r,s)$ dapat 
    direpresentasikan dalam satu elemen $g \in \vct{F,L,U,D,R,B}$ (grup Rubik) jika dan hanya jika:
    \begin{enumerate}
        \item $sgn(r)=sgn(s)$
        \item $sgn(r)=sgn(
            s) = 1 \iff u_1+u_2+\dots+u_6=0\Mod 2$
        \item $v_1+v_2+\dots+v_8 = 0\Mod 3$
        \item $w_1+w_2+\dots+w_{12} = 0\Mod 2$
    \end{enumerate}
\end{theorem}


\end{document}
