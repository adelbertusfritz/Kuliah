% ===== Setup Font =====
\usepackage[sfdefault,lf]{carlito}
\usepackage[T1]{fontenc}
\renewcommand*\oldstylenums[1]{\carlitoOsF #1}

% ==== Import Math Packages =====
\usepackage{amsmath, amssymb, amsthm}
\usepackage{mathtools}

\newtheorem{theorem}{Teorema}[section]
\newtheorem{corollary}{Akibat}[theorem]
\newtheorem{lemma}[theorem]{Lemma}
\newtheorem{definition}[theorem]{Definisi}

% ==== Import Styling Packages =====
\usepackage{enumitem}
\usepackage[pages=some, placement=bottom]{background}
\usepackage{moresize}
\usepackage{relsize}
\usepackage{hyperref}
\hypersetup{colorlinks=true,allcolors=blue}
\usepackage{hypcap}
\usepackage{verbatim}
\usepackage[normalem]{ulem}

\usepackage{hyperref}

% ==== Custom Declarations =====
\DeclarePairedDelimiter\abs{\lvert}{\rvert}
\DeclarePairedDelimiter\floor{\lfloor}{\rfloor}
\DeclarePairedDelimiter\cic{[ }{] }
\DeclarePairedDelimiter\oic{( }{] }
\DeclarePairedDelimiter\cio{[ }{) }
\DeclarePairedDelimiter\oio{( }{) }
\DeclarePairedDelimiter\set{\{ }{\} }
\DeclarePairedDelimiter\brk{(}{)}
\DeclarePairedDelimiter\vct{\langle}{\rangle}
\newcommand{\Mod}[1]{\ (\mathrm{mod}\ #1)}
\newcommand{\drv}[2]{\frac{d}{d#1}\brk*{#2}}
\newcommand{\drvL}[2]{D_{#1}\brk*{#2}}
\newcommand{\ds}{\displaystyle}
\newcommand{\eval}[3]{\left.\brk*{#1}\right\rvert_{#2}^{#3}}
\newcommand{\R}{\mathbb{R}}
\newcommand{\Rubik}{
    \draw[black, thick] (0,0) -- (2.598,1.5);
    \draw[black, thick] (0,0) -- (-2.598,1.5);
    \draw[black, thick] (0,0) -- (0,-3);
    \draw[black, thick] (0,-3) -- (2.598,-1.5);
    \draw[black, thick] (0,-3) -- (-2.598,-1.5);
    \draw[black, thick] (0,-2) -- (2.598,-0.5);
    \draw[black, thick] (0,-2) -- (-2.598,-0.5);
    \draw[black, thick] (0,-1) -- (2.598,0.5);
    \draw[black, thick] (0,-1) -- (-2.598,0.5);
    \draw[black, thick] (2.598,-1.5) -- (2.598,1.5);
    \draw[black, thick] (-2.598,-1.5) -- (-2.598,1.5);
    \draw[black, thick] (0,3) -- (2.598,1.5);
    \draw[black, thick] (0,3) -- (-2.598,1.5);
    \draw[black, thick] (0.867,0.5) -- (0.867,-2.5);
    \draw[black, thick] (1.732,1) -- (1.732,-2);
    \draw[black, thick] (-0.867,0.5) -- (-0.867,-2.5);
    \draw[black, thick] (-1.732,1) -- (-1.732,-2);
    \draw[black, thick] (-0.867,0.5) -- (1.732,2);
    \draw[black, thick] (-1.732,1) -- (0.867,2.5);
    \draw[black, thick] (0.867,0.5) -- (-1.732,2);
    \draw[black, thick] (1.732,1) -- (-0.867,2.5);
}

\newcommand{\RubikU}{
    \draw[black, thick] (0,0) -- (2.598,1.5);
    \draw[black, thick] (0,0) -- (-2.598,1.5);
    \draw[black, thick] (0,3) -- (0,6);
    \draw[black, thick] (0,6) -- (2.598,4.5);
    \draw[black, thick] (0,6) -- (-2.598,4.5);
    \draw[black, thick] (0,4) -- (2.598,2.5);
    \draw[black, thick] (0,4) -- (-2.598,2.5);
    \draw[black, thick] (0,5) -- (2.598,3.5);
    \draw[black, thick] (0,5) -- (-2.598,3.5);
    \draw[black, thick] (2.598,4.5) -- (2.598,1.5);
    \draw[black, thick] (-2.598,4.5) -- (-2.598,1.5);
    \draw[black, thick] (0,3) -- (2.598,1.5);
    \draw[black, thick] (0,3) -- (-2.598,1.5);
    \draw[black, thick] (0.867,5.5) -- (0.867,2.5);
    \draw[black, thick] (1.732,5) -- (1.732,2);
    \draw[black, thick] (-0.867,2.5) -- (-0.867,5.5);
    \draw[black, thick] (-1.732,2) -- (-1.732,5);
    \draw[black, thick] (-0.867,0.5) -- (1.732,2);
    \draw[black, thick] (-1.732,1) -- (0.867,2.5);
    \draw[black, thick] (0.867,0.5) -- (-1.732,2);
    \draw[black, thick] (1.732,1) -- (-0.867,2.5);
}