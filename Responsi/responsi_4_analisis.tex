% ===== Setup Page Layout =====
\documentclass{article}
\usepackage{geometry}
 \geometry{
 a4paper,
 total={15cm, 20cm},
 }
\usepackage{graphicx}
% ===== Setup Font =====
\usepackage[sfdefault,lf]{carlito}
\usepackage[T1]{fontenc}
\renewcommand*\oldstylenums[1]{\carlitoOsF #1}

% ==== Import Math Packages =====
\usepackage{amsmath, amssymb, amsthm}
\usepackage{mathtools}

% ==== Import Styling Packages =====
\usepackage{enumitem}
\usepackage[pages=some, placement=bottom]{background}
\usepackage{moresize}
\usepackage{relsize}
\usepackage{hyperref}
\hypersetup{colorlinks=true,allcolors=blue}
\usepackage{hypcap}
\usepackage{verbatim}
\usepackage[normalem]{ulem}

\usepackage{hyperref}

% ==== Custom Declarations =====
\DeclarePairedDelimiter\abs{\lvert}{\rvert}
\DeclarePairedDelimiter\floor{\lfloor}{\rfloor}
\DeclarePairedDelimiter\cic{[ }{] }
\DeclarePairedDelimiter\oic{( }{] }
\DeclarePairedDelimiter\cio{[ }{) }
\DeclarePairedDelimiter\oio{( }{) }
\DeclarePairedDelimiter\set{\{ }{\} }
\DeclarePairedDelimiter\brk{(}{)}
\newcommand{\drv}[2]{\frac{d}{d#1}\brk*{#2}}
\newcommand{\drvL}[2]{D_{#1}\brk*{#2}}
\newcommand{\ds}{\displaystyle}
\newcommand{\eval}[3]{\left.\brk*{#1}\right\rvert_{#2}^{#3}}
\newcommand{\R}{\mathbb{R}}
\newcommand{\vecu}{\textbf{u}}

\title{Responsi 4}
\author{Fritz Adelbertus Sitindaon}
\date{}

\begin{document}
\begin{flushright}
    \section*{Responsi 4 Analisis 1}
    \textbf{Tim Asisten Dosen}
\end{flushright}


\vspace{0.5cm}\hrule height 2pt\vspace{0.5cm}



\begin{center}
\textbf{\large{SOAL}}
\end{center}
\begin{enumerate}[leftmargin=*, label={\arabic*}.]
\item Buktikan untuk $a,b,c \in \mathbb{R}$ berlaku
\[
    \abs*{\sqrt{a^2+b^2}-\sqrt{a^2+c^2}} \leq \abs{b-c}
\]
\item Misalkan $S \subset \R$ adalah suatu himpunan tak kosong dan terbatas. Misalkan pula
\[
    T := \{as+b: s\in S\}, \text{dengan } a,b\in \R \text{ dan } a\neq 0
\]
Tentukanlah formula dari $\sup T$ yang bergantung pada $\sup S$ dan $\inf S$. Buktikan kebenaran formula tersebut.

\item Misalkan $\epsilon > 0$ dan $\delta > 0$. Jika $V_r(a)$ menyatakan lingkungan 
dari $a \in \R$ dengan radius $r$, tunjukkan bahwa $V_\epsilon(a)\cap V_\delta(a)$ 
dan $V_\epsilon(a) \cup V_\delta(a)$ merupakan lingkungan dari $a$ dengan raidus 
$\gamma$, untuk nilai $\gamma$ yang sesuai.

\item Diketahui interval bersarang dengan $I_n = \cio*{-3-\frac{1}{n!},\frac{3}{2}+\frac{2}{n^2}}$. 
Tentukan semua elemen dari $\bigcap_{n=1}^{\infty}I_n$ dan sertakan buktinya.

% \item Misalkan $a,b$ adalah bilangan real dengan $0 < a < b$. Gunakanlah teorema limit 
% untuk menunjukkan barisan $(x_n)=((a^n+b^n)^{1/n})$ adalah barisan yang konvergen 
% dan tentukan nilai limitnya.

% \item Diberikan suatu barisan $(x_n)$ yang didefinisikan dengan
% \[
%     x_1=\frac{3}{2}\quad x_{n+1}=\sqrt{3x_n-2}
% \]
% Buktikan barisan $x$ adalah barisan yang monoton dan terbatas.

% \item $I_n=\oio*{1,1+\frac{1}{n}}, n\in\mathbb{N}$, buktikan bahwa $\bigcap_{n=1}^{\infty} I_n$ itu kosong
% (Hint: Gunakan kontradiksi)\

\item Misalkan diberi barisan konvergen $(x_n)$ dan $(y_n)$. Didefinisikan barisan
$(u_n)$ dengan $u_n=\frac{x_n+y_n}{3}$. Tunjukkan barian $(u_n)$ konvergen dengan 
menggunakan definisi barisan konvergen. Jelaskan!

\item Misalkan $x_n := 1/1^2+1/2^2+\dots+1/n^2$ untuk setiap $n \in \mathbb{N}$. Buktikan $(x_n)$ konvergen dengan
membuktikan $(x_n)$ monoton naik dan terbatas. \\\textit{Hint:} jika $k \geq 2$, maka $1/k^2\leq 1/k(k-1)=1/(k-1)-1/k$,

\item Jika $\lim(x_n)=x > 0$, buktikan terdapat $K \in \mathbb{N}$ sehingga $\forall n \geq K$,
\[
    \frac{1}{2}x < x_n < 2x
\]
\end{enumerate}

\vspace{0.2cm}\hrule height 1pt


\newpage
\begin{center}
\textbf{\large{PEMBAHASAN}}
\end{center}
\begin{enumerate}[leftmargin=*, label={\arabic*}.]
\item Perhatikan \begin{align*}
    &\abs*{\sqrt{a^2+b^2}-\sqrt{a^2+c^2}}\\
    =&\abs*{\frac{b^2-c^2}{\sqrt{a^2+b^2}+\sqrt{a^2+c^2}}}\\
    \leq&\abs*{\frac{b^2-c^2}{\sqrt{b^2}+\sqrt{c^2}}}\\
    =&\abs*{\frac{b^2-c^2}{\abs{b}+\abs{c}}}\\
    \leq&\abs*{\frac{b^2-c^2}{b+c}}\\
    =&\abs{b-c}
\end{align*}
\item Karena $S$ terbatas maka
\begin{align*}
    &\inf S \leq s \leq \sup S\\
    \iff &a \inf S + b \leq as+b \leq a \sup S + b &\text{saat $a>0$}\\
    \text{atau } &a \inf S + b \geq as+b \geq a \sup S + b &\text{saat $a<0$}
\end{align*}
Berlaku untuk semua $s$, maka $T=\{as+b, s\in S\}$ terbatas


\[\sup T=
\begin{cases}
    a \sup S + b, &\text{jika $a > 0$}\\
    a \inf S + b, &\text{jika $a < 0$}    
\end{cases}\]
Bukti: \\
Kasus $a > 0$\\
Asumsikan $a \sup S + b$ bukan $\sup T$. maka
\begin{align*}
    &a \sup S + b \text{ bukan batas atas terkecil dari $as+b$}\\
    \iff&a \sup S \text{ bukan batas atas terkecil dari $as$}\\
    \iff&\sup S \text{ bukan batas atas terkecil dari $s$}
\end{align*}
Kontradiksi dengan definisi $\sup S$, maka haruslah $a \sup S + b = \sup T$

Kasus $a < 0$\\
Asumsikan $a \inf S + b$ bukan $\sup T$. maka
\begin{align*}
    &a \inf S + b \text{ bukan batas atas terkecil dari $as+b$}\\
    \iff&a \inf S \text{ bukan batas atas terkecil dari $as$}\\
    \iff&\inf S \text{ bukan batas bawah terkecil dari $s$}
\end{align*}
Kontradiksi dengan definisi $\inf S$, maka haruslah $a \inf S + b = \sup T$

\item Untuk $V_\epsilon (a)\cap V_\delta(a)$, Klaim: $\gamma = \min\{\epsilon, \delta\}$.\\
Ambil sembarang $x \in V_\epsilon (a)\cap V_\delta(a)$, dari definisi irisan himpunan $x \in V_\epsilon (a)$ dan $x \in V_\delta(a)$.\\
Dari definisi neigborhood:
\[
    |x-a| < \epsilon \text{ dan } |x-a| < \delta
\]
ini ekuivalen dengan 
\[
    |x-a| < \min\{\epsilon, \delta\}
\]
Maka $V_\epsilon (a)\cap V_\delta(a) = V_{\min\{\epsilon, \delta\}}(a)$\\

Untuk $V_\epsilon (a)\cup V_\delta(a)$, Klaim: $\gamma = \max\{\epsilon, \delta\}$.\\
Ambil sembarang $x \in V_\epsilon (a)\cup V_\delta(a)$, dari definisi gabungan himpunan $x \in V_\epsilon (a)$ atau $x \in V_\delta(a)$.\\
Dari definisi neigborhood:
\[
    |x-a| < \epsilon \text{ atau } |x-a| < \delta
\]
ini ekuivalen dengan 
\[
    |x-a| < \max\{\epsilon, \delta\}
\]
Maka $V_\epsilon (a)\cup V_\delta(a) = V_{\max\{\epsilon, \delta\}}(a)$

\item Klaim: $\bigcap_{n=1}^{\infty}I_n = [-3,\frac{3}{2}]$.
Ambil sembarang $x \in [-3,\frac{3}{2}]$ maka
\begin{align*}
    &-3 \leq x \leq \frac{3}{2}\\
    \iff&-3-\frac{1}{n!} < -3 \leq x \leq \frac{3}{2} < \frac{3}{2}+\frac{2}{n^2}\\
    \iff&-3-\frac{1}{n!} < x < \frac{3}{2}+\frac{2}{n^2}\\
    \iff&x \in \cio*{-3-\frac{1}{n!},\frac{3}{2}+\frac{2}{n^2}} = I_n
\end{align*}
Ini berlaku untuk semua $n\in \mathbb{N}$ sehingga untuk semua $x\in [-3,\frac{3}{2}], x\in\bigcap_{n=1}^{\infty}I_n$\\
Misalkan $x < -3$, adib $x\notin \bigcap_{n=1}^{\infty}I_n$\\
Asumsikan $x \in \bigcap_{n=1}^{\infty}I_n$, maka\\
\[-3-\frac{1}{n!} \leq x < \frac{3}{2}+\frac{2}{n^2}, \forall n \in \mathbb{N}\] 
dari persamaan kiri dan karena $x < - 3 \iff x+3 < 0$
\begin{align*}
    -3-\frac{1}{n!} \leq x &\iff -\frac{1}{n!} \leq x+3\\
    &\iff  -\frac{1}{n!} \leq x+3 < 0\\
    &\iff \frac{1}{n!} \geq -(x+3) > 0
\end{align*}
misalkan $y = -(x+3)$, karena $x+3 < 0$ maka $y = -(x+3) > 0$ sehingga
\[
    \frac{1}{n!} \geq -(x+3) > 0 \iff \frac{1}{n!} \geq y > 0 \iff \frac{1}{y} \geq n! > n > 0
\]
Ini berlaku untuk semua $n\in \mathbb{N}$, maka $\mathbb{N}$ terbatas oleh $\frac{1}{y}$, kontradiksi dengan Sifat Archimedes\\
Misalkan $x > \frac{3}{2}$, adib $x\notin \bigcap_{n=1}^{\infty}I_n$\\
Asumsikan $x \in \bigcap_{n=1}^{\infty}I_n$, maka\\
\[-3-\frac{1}{n!} < x < \frac{3}{2}+\frac{2}{n^2}, \forall n \in \mathbb{N}\] 
dari persamaan kanan dan karena $x > \frac{3}{2} \iff x-\frac{3}{2} > 0$
\begin{align*}
    x < \frac{3}{2}+\frac{2}{n^2} &\iff x - \frac{3}{2} < \frac{2}{n^2}\\
    &\iff  0 < x - \frac{3}{2} < \frac{2}{n^2}
\end{align*}
misalkan $z = x - \frac{3}{2}$, maka $z = x - \frac{3}{2} > 0$ sehingga
\[
    \frac{2}{n^2} > x - \frac{3}{2} > 0 \iff \frac{2}{n^2} > z > 0 \iff \frac{2}{z} > n^2 > n > 0
\]
Ini berlaku untuk semua $n\in \mathbb{N}$, maka $\mathbb{N}$ terbatas oleh $\frac{2}{z}$, kontradiksi dengan Sifat Archimedes\\
Dari semua bilangan riil, hanya kasus $x\in \cic*{3,\frac{3}{2}}$ yang memenuhi $x\in \bigcap_{n=1}^{\infty}I_n$.
Telah diperoleh semua nilai dari $\bigcap_{n=1}^{\infty}I_n$ yaitu $\cic*{3,\frac{3}{2}}$.
%\item Gunakan teorema apit
% \begin{align*}
%     &0 < a < b\\
%     \iff& 0 < a^n < b^n\\
%     \iff& b^n < a^n+b^n < 2b^n\\
%     \iff& b < \oio*{a^n+b^n}^{1/n} < 2^{1/n}b
% \end{align*}
% $\lim (b) = b$ dan $\lim (2^{1/n}b) = b$.

% \item Uji beberapa suku barisan:
% \[
%     x_1 = \frac{3}{2},\quad x_2=\sqrt{3x_1-2}=\sqrt{\frac{5}{2}} > \frac{3}{2} = x_1
% \]
% Klaim : $(x_n)$ monoton naik, dan $1 < x_n < 2, n\in\mathbb{N}$\\ 
% Akan dibuktikan $1 < x_n < 2$ dengan induksi.\\
% Base step: $1 < x_1 < 2$. Jelas karena $1 < \frac{3}{2} < 2$.\\
% Induksi: Asumsikan $1 < x_n < 2$, adib $1 < x_{n+1} < 2$.
% \begin{align*}
%     &1 < x_n < 2\\
%     \iff& 3 < 3x_n < 6\\
%     \iff& 1 < 3x_n-2 < 4\\
%     \iff& 1 < \sqrt{3x_n-2} < 2\\
%     \iff & 1 < x_{n+1} < 2.
% \end{align*}
% $(x_n)$ terbatas.\\
% Selanjutnya dibuktikan $(x_n)$ monoton naik, yaitu $x_{n+1}>x_n$.\\
% Dari pembuktian terbatas diperoleh $x_n > 1$, dan $2 > x_n$, maka $x_n-1>0$ dan $x_n-2 < 0$\\
% Sehingga
% \begin{align*}
%     &(x_n-1)(x_n-2) < 0\\
%     \iff&x_n^2-3x_n+2 < 0\\
%     \iff&x_n^2 < 3x_n-2\\
%     \iff&\abs{x_n} < \sqrt{3x_n-2}\\
%     \iff&x_n < x_{n+1}
% \end{align*}
% $(x_n)$ monoton naik\\
% $\therefore$ $(x_n)$ monoton naik dan terbatas.

% \item Asumsikan $\bigcap_{n=1}^{\infty} I_n$ tidak kosong,\\
% Maka terdapat $x\in \bigcap_{n=1}^{\infty} I_n$, yaitu $1 < x < 1+\frac{1}{n},\forall n\in\mathbb{N}$.\\
% Dari pertidaksamaan dikiri, diperoleh $x > 1$ sehingga $x \neq 1$ dan $x-1 > 0$ positif.\\
% Dari persamaan dikanan
% \begin{align*}
%     &x < 1+\frac{1}{n}, \forall n\in\mathbb{N}\\
%     \iff&x-1 < \frac{1}{n}, \forall n\in\mathbb{N}\\
%     \iff&n < \frac{1}{x-1}, \forall n\in\mathbb{N}\quad (*)\,x\neq 1, x-1 > 0
% \end{align*}
% Ini berarti bilangan natural terbatas oleh $\frac{1}{x-1}$, kontradiksi dengan Sifat Archimedes.\\
% Maka asumsi awal salah, haruslah $\bigcap_{n=1}^{\infty} I_n$ kosong.


\item $(x_n)$ dan $(y_n)$ konvergen, Misalkan $\lim(x_n)=x$ dan $\lim(y_n)=y$.\\
Klaim: $\lim(u_n) = \frac{x+y}{3}$.\\
Analisis Pendahuluan:\\
$\forall \epsilon > 0 \exists K?\in\mathbb{N} \ni \forall n > K$
\begin{align*}
    \abs*{u_n-\frac{x+y}{3}}&=\abs*{\frac{x_n+y_n}{3}-\frac{x+y}{3}}\\
    &=\frac{1}{3}\abs*{(x_n-x)+(y_n-y)}
\end{align*}
Karena $(x_n)$ dan $(y_n)$ konvergen, maka\\
\[\forall \epsilon > 0 \exists K_1\in\mathbb{N} \ni \forall k > K_1,\quad
\abs{x_k-x} < \epsilon\]
dan
\[\forall 2\epsilon > 0 \exists K_2\in\mathbb{N} \ni \forall l > K_1,\quad
\abs{x_l-x} < 2\epsilon\]
Dengan memilih $K =\max\{K_1,K_2\}$ maka untuk setiap $n > K$
\[
    \frac{1}{3}\abs{(x_n-x)+(y_n-x)} \leq \frac{1}{3}\oio*{\abs{x_n-x}+\abs{y_n-y}}<
    \frac{1}{3}\oio*{\epsilon+2\epsilon} = \epsilon
\]
Bukti Formal: $(u_n)$ konvergen karena
\begin{align*}
    \forall \epsilon > 0 \exists K = \max\{K_1,K_2\} \in \mathbb{N} \ni \forall n > K\\
    \abs*{u_n - \frac{x+y}{3}} \leq \frac{1}{3}\oio*{\abs{x_n-x}+\abs{y_n-y}} < \epsilon
\end{align*}

\item Barisan tersebut ekuivalen dengan definisi rekursif
\[
x_1 = 1,\quad x_{n+1} = x_n+\frac{1}{(n+1)^2}
\]
Uji beberapa titik:
\[
x_1 = 1, \quad x_2=x_1+\frac{1}{2^2} = \frac{5}{4}
\]
Klaim: $(x_n)$ monoton naik dan $x_n < 2, n \in \mathbb{N}$\\
Adib $(x_n)$ monoton naik, untuk sembarang $n \in \mathbb{N}$ jeals $n+1>0$ dan $\frac{1}{n+1} > 0$, maka
\[
\frac{1}{(n+1)^2} > 0 \iff x_n +\frac{1}{(n+1)^2} > x_n \iff x_{n+1} > x_n
\]
Adib $x_n < 2, n\in\mathbb{N}$, gunakan induksi matematik\\
Base step: Jelas $x_1 = 1 < 2$\\
Induction step: Asumsikan $x_n < 2$
\begin{align*}
    x_{n+1} &= x_n + \frac{1}{(n+1)^2}\\
    &\leq x_n + \frac{1}{n} - \frac{1}{n+1} &\text{(*) Gunakan Hint}\\
    &=x_{n-1} + \frac{1}{n^2} + \frac{1}{n} - \frac{1}{n+1}\\
    &\leq x_{n-1} + \frac{1}{n-1} - \frac{1}{n} + \frac{1}{n} - \frac{1}{n+1} &\text{(*) Gunakan Hint}\\
    &= x_{n_1} + \frac{1}{n-1} - \frac{1}{n+1}\\
    &\vdots\\
    &\leq x_2 + \frac{1}{2} - \frac{1}{n+1}\\
    &=\frac{5}{4} + \frac{1}{2} - \frac{1}{n+1} = \frac{7}{4} - \frac{1}{n+1} < \frac{7}{4} < 2
\end{align*}
Maka $(x_n)$ terbatas diatas. Karena $(x_n)$ mononton naik dan terbatas diatas, berdasarkan teorema konvergensi monoton 
barisan ini konvergen.

\item
Analisis Pendahuluan:\\
$\exists K?\in \mathbb{N} \ni \forall n \geq K, \quad \frac{1}{2}x < x_n < 2x$\\
Karena $\lim(x_n) = x > 0$ maka $\forall \epsilon > 0 \exists M \in \mathbb{N} \ni \forall n \geq M,\quad \abs{x_n-x} < \epsilon$\\
Pertama pilih $\epsilon = x > 0$, maka $\exists M_1 \in \mathbb{N} \ni \forall n \geq M_1$
\begin{align*}
    &{\abs{x_n-x} < x}\\
    \iff&-x < x_n-x < x\\
    \iff&0 < x_n < 2x\\
\end{align*}
Kedua pilih $\epsilon = \frac{1}{2}x > 0$, maka $\exists M_2 \in \mathbb{N} \ni \forall n \geq M_2$
\begin{align*}
    &{\abs{x_n-x} < \frac{1}{2}x}\\
    \iff&-\frac{1}{2}x < x_n-x < \frac{1}{2}x\\
    \iff&\frac{1}{2}< x_n < \frac{3}{2}x\\
\end{align*}
Ambil bagian kanan dan kiri yang sesuai sehingga
\[
\frac{1}{2}x < x_n < 2x
\]
Ini berlaku dengan memilih $K = \max\{M_1,M_2\}$.\\
Bukti Formal:\\
Terdapat $K = \max\{M_1,M_2\} \in \mathbb{N}$ dari $\lim(x_n)=x > 0$ sehingga $\forall n \geq K$,
\[
    \frac{1}{2}x < x_n < 2x
\]
\end{enumerate}

\end{document}
