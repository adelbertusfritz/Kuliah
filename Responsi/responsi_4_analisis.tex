% ===== Setup Page Layout =====
\documentclass{article}
\usepackage{geometry}
 \geometry{
 a4paper,
 total={15cm, 20cm},
 }
\usepackage{graphicx}
% ===== Setup Font =====
\usepackage[sfdefault,lf]{carlito}
\usepackage[T1]{fontenc}
\renewcommand*\oldstylenums[1]{\carlitoOsF #1}

% ==== Import Math Packages =====
\usepackage{amsmath, amssymb, amsthm}
\usepackage{mathtools}

\newtheorem{theorem}{Teorema}[section]
\newtheorem{corollary}{Akibat}[theorem]
\newtheorem{lemma}[theorem]{Lemma}
\newtheorem{definition}[theorem]{Definisi}

% ==== Import Styling Packages =====
\usepackage{enumitem}
\usepackage[pages=some, placement=bottom]{background}
\usepackage{moresize}
\usepackage{relsize}
\usepackage{hyperref}
\hypersetup{colorlinks=true,allcolors=blue}
\usepackage{hypcap}
\usepackage{verbatim}
\usepackage[normalem]{ulem}

\usepackage{hyperref}

% ==== Custom Declarations =====
\DeclarePairedDelimiter\abs{\lvert}{\rvert}
\DeclarePairedDelimiter\floor{\lfloor}{\rfloor}
\DeclarePairedDelimiter\cic{[ }{] }
\DeclarePairedDelimiter\oic{( }{] }
\DeclarePairedDelimiter\cio{[ }{) }
\DeclarePairedDelimiter\oio{( }{) }
\DeclarePairedDelimiter\set{\{ }{\} }
\DeclarePairedDelimiter\brk{(}{)}
\DeclarePairedDelimiter\vct{\langle}{\rangle}
\newcommand{\Mod}[1]{\ (\mathrm{mod}\ #1)}
\newcommand{\drv}[2]{\frac{d}{d#1}\brk*{#2}}
\newcommand{\drvL}[2]{D_{#1}\brk*{#2}}
\newcommand{\ds}{\displaystyle}
\newcommand{\eval}[3]{\left.\brk*{#1}\right\rvert_{#2}^{#3}}
\newcommand{\R}{\mathbb{R}}
\newcommand{\Rubik}{
    \draw[black, thick] (0,0) -- (2.598,1.5);
    \draw[black, thick] (0,0) -- (-2.598,1.5);
    \draw[black, thick] (0,0) -- (0,-3);
    \draw[black, thick] (0,-3) -- (2.598,-1.5);
    \draw[black, thick] (0,-3) -- (-2.598,-1.5);
    \draw[black, thick] (0,-2) -- (2.598,-0.5);
    \draw[black, thick] (0,-2) -- (-2.598,-0.5);
    \draw[black, thick] (0,-1) -- (2.598,0.5);
    \draw[black, thick] (0,-1) -- (-2.598,0.5);
    \draw[black, thick] (2.598,-1.5) -- (2.598,1.5);
    \draw[black, thick] (-2.598,-1.5) -- (-2.598,1.5);
    \draw[black, thick] (0,3) -- (2.598,1.5);
    \draw[black, thick] (0,3) -- (-2.598,1.5);
    \draw[black, thick] (0.867,0.5) -- (0.867,-2.5);
    \draw[black, thick] (1.732,1) -- (1.732,-2);
    \draw[black, thick] (-0.867,0.5) -- (-0.867,-2.5);
    \draw[black, thick] (-1.732,1) -- (-1.732,-2);
    \draw[black, thick] (-0.867,0.5) -- (1.732,2);
    \draw[black, thick] (-1.732,1) -- (0.867,2.5);
    \draw[black, thick] (0.867,0.5) -- (-1.732,2);
    \draw[black, thick] (1.732,1) -- (-0.867,2.5);
}

\newcommand{\RubikU}{
    \draw[black, thick] (0,0) -- (2.598,1.5);
    \draw[black, thick] (0,0) -- (-2.598,1.5);
    \draw[black, thick] (0,3) -- (0,6);
    \draw[black, thick] (0,6) -- (2.598,4.5);
    \draw[black, thick] (0,6) -- (-2.598,4.5);
    \draw[black, thick] (0,4) -- (2.598,2.5);
    \draw[black, thick] (0,4) -- (-2.598,2.5);
    \draw[black, thick] (0,5) -- (2.598,3.5);
    \draw[black, thick] (0,5) -- (-2.598,3.5);
    \draw[black, thick] (2.598,4.5) -- (2.598,1.5);
    \draw[black, thick] (-2.598,4.5) -- (-2.598,1.5);
    \draw[black, thick] (0,3) -- (2.598,1.5);
    \draw[black, thick] (0,3) -- (-2.598,1.5);
    \draw[black, thick] (0.867,5.5) -- (0.867,2.5);
    \draw[black, thick] (1.732,5) -- (1.732,2);
    \draw[black, thick] (-0.867,2.5) -- (-0.867,5.5);
    \draw[black, thick] (-1.732,2) -- (-1.732,5);
    \draw[black, thick] (-0.867,0.5) -- (1.732,2);
    \draw[black, thick] (-1.732,1) -- (0.867,2.5);
    \draw[black, thick] (0.867,0.5) -- (-1.732,2);
    \draw[black, thick] (1.732,1) -- (-0.867,2.5);
}

\title{Responsi 4}
\author{Fritz Adelbertus Sitindaon}
\date{}

\begin{document}
\begin{flushright}
    \section*{Responsi 4 Analisis 1}
    \textbf{Tim Asisten Dosen}
\end{flushright}


\vspace{0.5cm}\hrule height 2pt\vspace{0.5cm}



\begin{center}
\textbf{\large{SOAL}}
\end{center}
\begin{enumerate}[leftmargin=*, label={\arabic*}.]
\item Misalkan $S \subset \R$ adalah suatu himpunan tak kosong dan terbatas. Misalkan pula
\[
    T := \{as+b: s\in S\}, \text{dengan } a,b\in \R \text{ dan } a\neq 0
\]
Tentukanlah formula dari $\sup T$ yang bergantung pada $\sup S$ dan $\inf S$. Buktikan kebenaran formula tersebut.

\item Misalkan $\epsilon > 0$ dan $\delta > 0$. Jika $V_r(a)$ menyatakan lingkungan 
dari $a \in \R$ dengan radius $r$, tunjukkan bahwa $V_\epsilon(a)\cap V_\delta(a)$ 
dan $V_\epsilon(a) \cup V_\delta(a)$ merupakan lingkungan dari $a$ dengan raidus 
$\gamma$, untuk nilai $\gamma$ yang sesuai.

\item Misalkan $a,b$ adalah bilangan real dengan $0 < a < b$. Gunakanlah teorema limit 
untuk menunjukkan barisan $(x_n)=((a^n+b^n)^{1/n})$ adalah barisan yang konvergen 
dan tentukan nilai limitnya.
\end{enumerate}


\vspace{0.2cm}\hrule height 1pt


\newpage
\begin{center}
\textbf{\large{PEMBAHASAN}}
\end{center}
\begin{enumerate}[leftmargin=*, label={\arabic*}.]
\item Karena $S$ terbatas maka
\begin{align*}
    &\inf S \leq s \leq \sup S\\
    \iff &a \inf S + b \leq as+b \leq a \sup S + b &\text{saat $a>0$}\\
    \text{atau } &a \inf S + b \geq as+b \geq a \sup S + b &\text{saat $a<0$}
\end{align*}
Berlaku untuk semua $s$, maka $T=\{as+b, s\in S\}$ terbatas


\[\sup T=
\begin{cases}
    a \sup S + b, &\text{jika $a > 0$}\\
    a \inf S + b, &\text{jika $a < 0$}    
\end{cases}\]
Bukti: Asumsikan bukan batas atas terkecil namun dari penjelasan sebelumnya 
$\sup T$ yang diasumsikan adalah batas atas. Nanti berujung $\sup S$ 
bukan batas atas terkecil $S$ atau $\inf S$ bukan batas bawah terbesar $S$. 
Tapi ini berkontradiksi dengan definisi $\sup S$ dan $\inf S$.

\item Untuk $V_\epsilon (a)\cap V_\delta(a)$, Klaim: $\gamma = \min\{\epsilon, \delta\}$.\\
Ambil sembarang $x \in V_\epsilon (a)\cap V_\delta(a)$, by definition $x \in V_\epsilon (a)$ dan $x \in V_\delta(a)$.\\
Dari definisi neigborhood:
\[
    |x-a| < \epsilon \text{ dan } |x-a| < \delta
\]
ini ekuivalen dengan 
\[
    |x-a| < \min\{\epsilon, \delta\}
\]
Maka $V_\epsilon (a)\cap V_\delta(a) = V_{\min\{\epsilon, \delta\}}(a)$\\

Untuk $V_\epsilon (a)\cup V_\delta(a)$, Klaim: $\gamma = \max\{\epsilon, \delta\}$.\\
Ambil sembarang $x \in V_\epsilon (a)\cup V_\delta(a)$, by definition $x \in V_\epsilon (a)$ atau $x \in V_\delta(a)$.\\
Dari definisi neigborhood:
\[
    |x-a| < \epsilon \text{ atau } |x-a| < \delta
\]
ini ekuivalen dengan 
\[
    |x-a| < \max\{\epsilon, \delta\}
\]
Maka $V_\epsilon (a)\cup V_\delta(a) = V_{\max\{\epsilon, \delta\}}(a)$

\item Gunakan teorema apit
\begin{align*}
    &0 < a < b\\
    \iff& 0 < a^n < b^n\\
    \iff& b^n < a^n+b^n < 2b^n\\
    \iff& b < \oio*{a^n+b^n}^{1/n} < 2^{1/n}b
\end{align*}
$\lim (b) = b$ dan $\lim (2^{1/n}b) = b$.

\end{enumerate}
\end{document}
