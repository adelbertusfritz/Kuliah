% ===== Setup Page Layout =====
\documentclass{article}
\usepackage{geometry}
 \geometry{
 a4paper,
 total={15cm, 20cm},
 }
\usepackage{graphicx}
% ===== Setup Font =====
\usepackage[sfdefault,lf]{carlito}
\usepackage[T1]{fontenc}
\renewcommand*\oldstylenums[1]{\carlitoOsF #1}

% ==== Import Math Packages =====
\usepackage{amsmath, amssymb, amsthm}
\usepackage{mathtools}

\newtheorem{theorem}{Teorema}[section]
\newtheorem{corollary}{Akibat}[theorem]
\newtheorem{lemma}[theorem]{Lemma}
\newtheorem{definition}[theorem]{Definisi}

% ==== Import Styling Packages =====
\usepackage{enumitem}
\usepackage[pages=some, placement=bottom]{background}
\usepackage{moresize}
\usepackage{relsize}
\usepackage{hyperref}
\hypersetup{colorlinks=true,allcolors=blue}
\usepackage{hypcap}
\usepackage{verbatim}
\usepackage[normalem]{ulem}

\usepackage{hyperref}

% ==== Custom Declarations =====
\DeclarePairedDelimiter\abs{\lvert}{\rvert}
\DeclarePairedDelimiter\floor{\lfloor}{\rfloor}
\DeclarePairedDelimiter\cic{[ }{] }
\DeclarePairedDelimiter\oic{( }{] }
\DeclarePairedDelimiter\cio{[ }{) }
\DeclarePairedDelimiter\oio{( }{) }
\DeclarePairedDelimiter\set{\{ }{\} }
\DeclarePairedDelimiter\brk{(}{)}
\DeclarePairedDelimiter\vct{\langle}{\rangle}
\newcommand{\Mod}[1]{\ (\mathrm{mod}\ #1)}
\newcommand{\drv}[2]{\frac{d}{d#1}\brk*{#2}}
\newcommand{\drvL}[2]{D_{#1}\brk*{#2}}
\newcommand{\ds}{\displaystyle}
\newcommand{\eval}[3]{\left.\brk*{#1}\right\rvert_{#2}^{#3}}
\newcommand{\R}{\mathbb{R}}
\newcommand{\Rubik}{
    \draw[black, thick] (0,0) -- (2.598,1.5);
    \draw[black, thick] (0,0) -- (-2.598,1.5);
    \draw[black, thick] (0,0) -- (0,-3);
    \draw[black, thick] (0,-3) -- (2.598,-1.5);
    \draw[black, thick] (0,-3) -- (-2.598,-1.5);
    \draw[black, thick] (0,-2) -- (2.598,-0.5);
    \draw[black, thick] (0,-2) -- (-2.598,-0.5);
    \draw[black, thick] (0,-1) -- (2.598,0.5);
    \draw[black, thick] (0,-1) -- (-2.598,0.5);
    \draw[black, thick] (2.598,-1.5) -- (2.598,1.5);
    \draw[black, thick] (-2.598,-1.5) -- (-2.598,1.5);
    \draw[black, thick] (0,3) -- (2.598,1.5);
    \draw[black, thick] (0,3) -- (-2.598,1.5);
    \draw[black, thick] (0.867,0.5) -- (0.867,-2.5);
    \draw[black, thick] (1.732,1) -- (1.732,-2);
    \draw[black, thick] (-0.867,0.5) -- (-0.867,-2.5);
    \draw[black, thick] (-1.732,1) -- (-1.732,-2);
    \draw[black, thick] (-0.867,0.5) -- (1.732,2);
    \draw[black, thick] (-1.732,1) -- (0.867,2.5);
    \draw[black, thick] (0.867,0.5) -- (-1.732,2);
    \draw[black, thick] (1.732,1) -- (-0.867,2.5);
}

\newcommand{\RubikU}{
    \draw[black, thick] (0,0) -- (2.598,1.5);
    \draw[black, thick] (0,0) -- (-2.598,1.5);
    \draw[black, thick] (0,3) -- (0,6);
    \draw[black, thick] (0,6) -- (2.598,4.5);
    \draw[black, thick] (0,6) -- (-2.598,4.5);
    \draw[black, thick] (0,4) -- (2.598,2.5);
    \draw[black, thick] (0,4) -- (-2.598,2.5);
    \draw[black, thick] (0,5) -- (2.598,3.5);
    \draw[black, thick] (0,5) -- (-2.598,3.5);
    \draw[black, thick] (2.598,4.5) -- (2.598,1.5);
    \draw[black, thick] (-2.598,4.5) -- (-2.598,1.5);
    \draw[black, thick] (0,3) -- (2.598,1.5);
    \draw[black, thick] (0,3) -- (-2.598,1.5);
    \draw[black, thick] (0.867,5.5) -- (0.867,2.5);
    \draw[black, thick] (1.732,5) -- (1.732,2);
    \draw[black, thick] (-0.867,2.5) -- (-0.867,5.5);
    \draw[black, thick] (-1.732,2) -- (-1.732,5);
    \draw[black, thick] (-0.867,0.5) -- (1.732,2);
    \draw[black, thick] (-1.732,1) -- (0.867,2.5);
    \draw[black, thick] (0.867,0.5) -- (-1.732,2);
    \draw[black, thick] (1.732,1) -- (-0.867,2.5);
}

\title{Responsi 8}
\author{Fritz Adelbertus Sitindaon}
\date{}

\begin{document}
\begin{flushright}
    \section*{Responsi 8 Analisis 1}
    \textbf{Tim Asisten Dosen}
\end{flushright}


\vspace{0.5cm}\hrule height 2pt\vspace{0.5cm}



% \begin{center}
% \textbf{\large{SOAL}}
% \end{center}
% \begin{enumerate}[leftmargin=*, label={\arabic*}.]
% \item Buktikan bahwa limit dari fungsi berikut ada. Gunakan definisi 
% limit fungsi untuk menjelaskan jawaban kalian.
% \[
%     \lim_{x\to 4}\frac{4-x}{2-\sqrt{x}}
% \]

% \item Buktikan barisan $(x_n)$ berikut adalah barisan kontraktif.
% \[
%     0 < x_1 < 1, \quad x_{n+1}=\frac{1}{10}(x_n^5+1),\, n\geq 1
% \]

% \item Misalkan $f$ adalah fungsi bernilai real di $(0,1)$ sedemikian sehingga $\ds\lim_{x\to 0}f(x)=0$
% \\dan $\ds\lim_{x\to 0}\frac{\oio*{f(x)-f(\frac{x}{2})}}{x}=0$
% \begin{enumerate}
%     \item Tunjukkanlah bahwa untuk $\epsilon > 0$ dan bilangan asli $n$, berlaku
%     \[
%     \abs*{f(x)-f\oio*{\frac{x}{2^n}}} < 2\epsilon x
%     \]
%     \item Hitunglah nilai dari $\ds\lim_{x\to 0}\frac{f(x)}{x}$
% \end{enumerate}

% \item Buktikan bahwa barisan $\oio*{\frac{\sqrt{2n^2+1}}{\sqrt{n}}}$
% itu divergen sejati.

% \item Misalkan fungsi $f\colon \R \to \R$ kontinu pada $\R$ sedemikian sehingga $f(x)=0$ untuk
% semua bilangan irasional $x$. Apakah $f(x)=0, \forall x\in\R$? Jelaskan jawaban Anda!

% \item Misalkan $f\colon \R \to \R$ adalah fungsi yang didefinisikan sebagai
% \[
%     f(x):=\begin{cases}
%         \abs{3x}, &\text{jika $x$ rasional}\\
%         x+8, &\text{jika $x$ irasional}
%     \end{cases}
% \]
% Tentukan titik-titik dimana fungsi $f$ kontinu dan titik-titik dimana fungsi $f$ diskontinu.
% Jelaskan jawaban kalian. 

% \end{enumerate}

% \vspace{0.2cm}\hrule height 1pt


% \newpage
\begin{center}
\textbf{\large{PEMBAHASAN}}
\end{center}
\begin{enumerate}[leftmargin=*, label={\arabic*}.]
\item Coretan: (Tebak nilai limit)
\[
\lim_{x\to 4} \frac{4-x}{2-\sqrt{x}}=
\lim_{x\to 4} \frac{(2-\sqrt{x})(2+\sqrt{x})}{2-\sqrt{x}}
\lim_{x\to 4} (2+\sqrt{x}) = 4
\]
Pembuktian: Adib $\lim_{x\to 4} \frac{4-x}{2-\sqrt{x}}=4$\\
Analisis Pendahuluan:\\
Ambil sembarang $\epsilon > 0, \exists \delta > 0? \ni \forall x, 0 < \abs{x-4}<\delta$ 
\begin{align*}
    \abs*{\frac{4-x}{2-\sqrt{x}}-4} &= \abs*{\frac{4-x-8+4\sqrt{x}}{2-\sqrt{x}}}
    \\&=\abs*{\frac{4-4\sqrt{x}+x}{2-\sqrt{x}}}
    \\&=\abs*{\frac{(2-\sqrt{x})(2-\sqrt{x})}{2-\sqrt{x}}}
    \\&=\abs*{(2-\sqrt{x})}\abs*{\frac{2+\sqrt{x}}{2+\sqrt{x}}}
    \\&=\abs*{\frac{4-x}{2+\sqrt{x}}}
    \\&\leq\abs*{\frac{x-4}{2}} < \frac{\delta}{2}
\end{align*}
Sehingga pilih $\delta = 2\epsilon$\\
Bukti Formal: $\lim_{x\to 4} \frac{4-x}{2-\sqrt{x}}=4$ karena\\
$\forall \epsilon > 0, \exists \delta = 2\epsilon > 0 \ni \forall x, 0 < \abs{x-4} < \delta,$
\[
    \abs*{\frac{4-x}{2-\sqrt{x}}-4} \leq\abs*{\frac{x-4}{2}} < \frac{\delta}{2} = \epsilon
\]
\item Adib barisan $(x_n)$ berikut adalah barisan kontraktif.
\[
    0 < x_1 < 1, \quad x_{n+1}=\frac{1}{10}(x_n^5+1),\, n\geq 1
\]
Pertama akan dibuktikan $0 < x_n < 1$ untuk semua $n \in \N$ dengan induksi matematik\\
Bukti:\\
Base ($n=1$): Jelas dari soal $0 < x_1 < 1$\\
Induksi: Asumsikan $0 < x_n < 1$, adib $0 < x_{n+1} < 1$\\
Perhatikan
\begin{align*}
    0 < x_n < 1 & \iff 0 < x_n^5 < 1\\
    & \iff 0 < x_n^5 + 1 < 2\\
    & \iff 0 < \frac{1}{10}(x_n^5 + 1) < \frac{1}{5} < 1\\
\end{align*}
Terbukti $0 < x_n < 1$ untuk $n \in \N$\\
Adib $(x_n)$ kontraktif.\\
Perhatikan lagi
\begin{align*}
    |x_{n+2} - x_{n+1}| &= \abs*{\frac{1}{10}(x_{n+1}^5+1)-\frac{1}{10}(x_n^5+1)}\\
    &= \frac{1}{10}\abs*{x_{n+1}^5+1-x_n^5-1}\\
    &= \frac{1}{10}\abs*{x_{n+1}^5-x_n^5}\\
    &< \frac{1}{10}\abs{5}\abs*{x_{n+1}-x_n}\quad (**)\\
    &= \frac{1}{2}\abs*{x_{n+1}-x_n}
\end{align*}
Maka $(x_n)$ barisan kontraktif dengan $C=\frac{1}{2}$
\begin{align*}
    (**) &\abs*{x_{n+1}^5-x_n^5}\\
    =\, & \abs{x_{n+1}-x_n}\abs{x_{n+1}^4+x_{n+1}^3x_n+x_{n+1}^2x_n^2+x_{n+1}x_n^3+x_n^4}\\
    &\text{ingat $0 < x_n < 1$ untuk semua $n \in \N$}\\
    <\, & \abs{x_{n+1}-x_n}\abs{1+1+1+1+1} = |5|\abs{x_{n+1}-x_n}
\end{align*}

\item Misalkan $f$ adalah fungsi bernilai real di $(0,1)$ sedemikian sehingga $\ds\lim_{x\to 0}f(x)=0$
\\dan $\ds\lim_{x\to 0}\frac{\oio*{f(x)-f(\frac{x}{2})}}{x}=0$
\begin{enumerate}
    \item Adib untuk $\epsilon > 0$ dan $n \in \N$, berlaku
    \[\abs*{f(x)-f\oio*{\frac{x}{2^n}}} < 2\epsilon x\]
    Ambil sembarang $\epsilon > 0$. Karena $\ds\lim_{x\to 0}\frac{\oio*{f(x)-f(\frac{x}{2})}}{x}=0$ maka
    \begin{align*}
        &\abs*{\frac{\oio*{f(x)-f(\frac{x}{2})}}{x}} < \epsilon\\
        \iff &\abs*{f(x)-f\oio*{\frac{x}{2}}} < \epsilon\abs{x}\\
        \iff &\abs*{f(x)-f\oio*{\frac{x}{2}}} < \epsilon x \quad\text{Karena $x>0$}\\
        &\text{Subtitusi x dengan x/2 diperoleh} \\
        &\abs*{f\oio*{\frac{x}{2}}-f\oio*{\frac{x}{4}}} < \frac{\epsilon x}{2} \\
        &\text{Sehingga secara umum}\\
        &\abs*{f\oio*{\frac{x}{2^n}}-f\oio*{\frac{x}{2^{n+1}}}} < \frac{\epsilon x}{2^n} 
    \end{align*} 
    Sekarang perhatikan bahwa
    \begin{align*}
        &\abs*{f(x)-f\oio*{\frac{x}{2^n}}}\\ 
        =\,&  \abs*{f(x)-f\oio*{\frac{x}{2}}+f\oio*{\frac{x}{2}}-f\oio*{\frac{x}{2^n}}} \\
        =\,&\abs*{f(x)-f\oio*{\frac{x}{2}}+f\oio*{\frac{x}{2}}+\dots-f\oio*{\frac{x}{2^{n-1}}}+f\oio*{\frac{x}{2^{n-1}}}-f\oio*{\frac{x}{2^n}}} \\
        \leq\,&\abs*{f(x)-f\oio*{\frac{x}{2}}}+\abs*{f\oio*{\frac{x}{2}}-f\oio*{\frac{x}{4}}}+\dots+\abs*{f\oio*{\frac{x}{2^{n-1}}}-f\oio*{\frac{x}{2^n}}}\\
        \leq\,& \epsilon x + \frac{ex}{2} + \dots + \frac{ex}{2^{n-1}}\\
        =\,&\epsilon x(1+\frac{1}{2}+\dots+\frac{1}{2^{n-1}})\\
        \leq\,& \epsilon x \sum_{k=0}^{\infty}\frac{1}{2^k} = \epsilon x \frac{1}{1-\frac{1}{2}} = 2\epsilon x
    \end{align*}
    Terbukti bahwa $\ds\abs*{f(x)-f\oio*{\frac{x}{2^n}}} < 2\epsilon x$
    \item Akan dihitung nilai dari $\lim_{x\to 0}\frac{f(x)}{x}$
    \\Setelah memperoleh jawaban sebelumnya, pandang barisan $(x_n)$ dengan
    \[
        x_n = \abs*{f(x)-f\oio*{\frac{x}{2^n}}}
    \]
    dan terbukti bahwa $\abs*{f(x)-f\oio*{\frac{x}{2^n}}} < 2\epsilon x$\\
    sehingga $\lim\oio*{\abs*{f(x)-f\oio*{\frac{x}{2^n}}}} < \lim\oio*{2\epsilon x}$\\
    perhatikan saat $n \to \infty$, $\frac{x}{2^n} \to 0$\\
    karena $\lim_{x\to 0} f(x)=0$,
    berdasarkan kriteria limit barisan, $\oio*{f\oio*{\frac{x}{2^n}}} \to 0$.\\
    Sehingga $\lim\oio*{\abs*{f(x)-f\oio*{\frac{x}{2^n}}}} = |f(x)| < \lim\oio*{2\epsilon x} = 2\epsilon x$\\
    Diperoleh $|f(x)| < 2ex \iff \abs*{\frac{f(x)}{x}} < 2e$.\\
    Dengan memilih $\delta = \frac{\epsilon}{2}$ maka $\lim_{x\to 0}\frac{f(x)}{x}=0$\\
    $\therefore$ Berdasarkan analisa ini nilai $\lim_{x\to 0}\frac{f(x)}{x}=0$
\end{enumerate}
\item Adib barisan $\oio*{\frac{\sqrt{2n^2+1}}{\sqrt{n}}}$ divergen sejati.

Analisis Pendahuluan:\\
Ambil sembarang $\alpha \in \R$. Akan dicari $K(\alpha)\in \N$ \\
sehingga $\forall n \geq K(\alpha), \frac{\sqrt{2n^2+1}}{\sqrt{n}} > \alpha$\\
\\Perhatikan
\begin{align*}
    \frac{\sqrt{2n^2+1}}{\sqrt{n}} > \frac{\sqrt{2n^2}}{\sqrt{n}} > \frac{\sqrt{n^2}}{\sqrt{n}} = \sqrt{n} > \sqrt{K(\alpha)} > \alpha
\end{align*}
Sehingga pilih $K(\alpha) > \alpha^2$ (Dijamin Sifat Archimedes)
\\Bukti Formal: $\oio*{\frac{\sqrt{2n^2+1}}{\sqrt{n}}}$ divergen sejati karena \\
$\forall \alpha > \R, \exists K(\alpha) > \alpha^2 \in \N$ (Dijamin Sifat Archimedes) sehingga
\\$\forall n \geq K(\alpha), \ds \frac{\sqrt{2n^2+1}}{\sqrt{n}} > \sqrt{n} > \sqrt{K(\alpha)} > \alpha$

\item Misalkan fungsi $f\colon \R \to \R$ kontinu pada $\R$ sedemikian sehingga $f(x)=0$ untuk
semua bilangan irasional $x$.

Klaim: $f(x)=0, \forall x \in \R$.\\
Bukti: (Dengan kontradiksi) Andaikan bahwa klaim salah.
\begin{itemize}
    \item Maka ada $c \in \R$, sehingga $f(c)\neq 0$.
    \item Karena $f(x)=0$ untuk semua $x$ irasional, $c$ adalah bilangan rasional.
    \item Kontruksi sebuah barisan $(x_n)$ terdiri dari bilangan irasional yg konvergen ke $c$ (Dijamin Density Theorem).
    \item Maka barisan $(f(x_n))$ konvergen ke $0$.
    \item Disisi lain $f$ kontinu maka dengan kriteria barisan kontinu barisan $(f(x_n))$ konvergen ke $f(c)$.
    \item Sehingga $f(c) = \lim((f(x_n))) = 0$.
    \item Ini berkontradiksi dengan poin pertama.
    \item Pengandaian salah maka klaim benar.
\end{itemize}



\item Misalkan $f\colon \R \to \R$ adalah fungsi yang didefinisikan sebagai
\[
    f(x):=\begin{cases}
        \abs{3x}, &\text{jika $x$ rasional}\\
        x+8, &\text{jika $x$ irasional}
    \end{cases}
\]
Akan ditentukan titik kontinu (Coretan):\\
Titik kontinu terjadi ketika $f(x)$ bernilai sama untuk $x$ rasional dan irasional.\\
Sehingga
\begin{align*}
    |3x| = x+8 \iff -3x = x+8 \text{ atau } 3x = x+8 \iff x = -2 \text{ atau } x=4
\end{align*}
Pembuktian:\\
Akan dibuktikan $f$ kontinu di $x=-2$ ($f(-2)=6$).\\
Analisis Pendahuluan:\\
Ambil sembarang $\epsilon > 0$, akan dicari $\delta$ sehingga $\forall x, \abs{x+2}<\delta$,\\
Untuk $x$ rasional
\begin{align*}
    |f(x)-6| &= \abs{\abs{3x}-6} \\
    &= \abs{\abs{3x}-\abs{-6}}\\
    &\leq |3x-(-6)|\\
    &= |3x+6|\\
    &=3|x+2|
\end{align*}
untuk $x$ rasional pilih $\delta = \epsilon/3$.

Sekarang untuk $x$ irasional
\begin{align*}
    |f(x)-6| &= \abs{x+8-6}=|x+2|
\end{align*}
untuk $x$ irasional pilih $\delta = \epsilon$.

Sehingga untuk $x \in \R$, pilih $\delta = \min\set{\epsilon,\epsilon/3}=\epsilon/3$.\\
Bukti Formal: $f$ kontinu di $x=-2$ karena\\
$\forall \epsilon > 0, \exists \delta = \epsilon/3 \ni \forall x, \abs{x+2} < \delta, \abs{f(x)-6} < \epsilon$

Akan dibuktikan $f$ kontinu di $x=4$ ($f(4)=12$).\\
Analisis Pendahuluan:\\
Ambil sembarang $\epsilon > 0$, akan dicari $\delta$ sehingga $\forall x, \abs{x-4}<\delta$,\\
Untuk $x$ rasional
\begin{align*}
    |f(x)-12| &= \abs{\abs{3x}-12} \\
    &= \abs{\abs{3x}-\abs{12}}\\
    &\leq |3x-12|\\
    &=3|x-4|
\end{align*}
untuk $x$ rasional pilih $\delta = \epsilon/3$.

Sekarang untuk $x$ irasional
\begin{align*}
    |f(x)-12| &= \abs{x+8-12}=|x-4|
\end{align*}
untuk $x$ irasional pilih $\delta = \epsilon$.

Sehingga untuk $x \in \R$, pilih $\delta = \min\set{\epsilon,\epsilon/3}=\epsilon/3$.\\
Bukti Formal: $f$ kontinu di $x=4$ karena\\
$\forall \epsilon > 0, \exists \delta = \epsilon/3 \ni \forall x, \abs{x-4} < \delta, \abs{f(x)-12} < \epsilon$

Akan dibuktikan $f$ tidak kontinu di $c$ rasional dengan $c\neq -2$ dan $c\neq 4$.\\
Bukti: Gunakan kriteria diskontinuitas.
\begin{itemize}
    \item Konstruksi barisan $(x_n)$ yang terdiri dari bilangan irasional yang konvergen ke $c$. (Dijamin Teorema Densitas).
    \item Maka barisan $(f(x_n))$ konvergen ke $c+8$
    \item Tetapi $f(c) = |3c|$ dan karena $c\neq -2$ dan $c\neq 4$ maka $f(c) = |3x| \neq c+8 = \lim((f(x_n)))$
    \item Berdasarkan kriteria diskontinuitas, maka $f$ tidak kontinu di $c$ rasional dengan $c\neq -2$ dan $c\neq 4$.
\end{itemize}

Akan dibuktikan $f$ tidak kontinu di $c$ irasional.\\
Bukti: Gunakan kriteria diskontinuitas.
\begin{itemize}
    \item Konstruksi barisan $(x_n)$ yang terdiri dari bilangan rasional yang konvergen ke $c$. (Dijamin Teorema Densitas).
    \item Maka barisan $(f(x_n))$ konvergen ke $|3c|$
    \item Tetapi $f(c) = c+8$ dan $|3c| = c+8$ saat $c=-2$ atau $c=4$ (rasional)
    \item $c$ irasional sehingga $f(c) = c+8 \neq |3c| = \lim(f(x_n))$.
    \item Berdasarkan kriteria diskontinuitas, maka $f$ tidak kontinu di $c$ irasional.
\end{itemize}
\end{enumerate}

\end{document}
